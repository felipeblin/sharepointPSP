\documentclass[12pt]{article}
\usepackage[spanish]{babel}
\usepackage[utf8]{inputenc}
\usepackage[T1]{fontenc}
\usepackage{lmodern}
\usepackage[a4paper,margin=2.5cm]{geometry}
\usepackage{xcolor}
\title{\textbf{Glosario de Términos por Área}}
\author{Revista Científica de Gestión Inmobiliaria}
\date{\today}
\begin{document}
\maketitle
\tableofcontents
\section*{Presentación}
\addcontentsline{toc}{section}{Presentación}
Este documento es un Glosario de Términos por Área. 
\newpage
\section{\textcolor{blue}{Comercial}}
\subsection{\textcolor{teal}{Comercial}:{Acta de entrega al cliente}}
\textbf{Definición:} Documento que se genera cuando las condiciones del departamento han sido revisadas y no se encontraron observaciones. Si no hay observaciones, el cliente firma el acta.
\subsection{\textcolor{teal}{Comercial}:{Acta de lectura de medidores}}
\textbf{Definición:} Documento que registra la lectura de medidores (como agua, electricidad, gas) en el momento de la entrega.
\subsection{\textcolor{teal}{Comercial}:{Administración de gastos comunes}}
\textbf{Definición:} La gestión y control de los gastos comunes de la comunidad hasta que se enajene un porcentaje mayoritario de las propiedades (aprox. 70%). Incluye la distribución equitativa de los costos entre los propietarios y la supervisión de los fondos para mantener las instalaciones y servicios comunes.
\subsection{\textcolor{teal}{Comercial}:{Administración de la comunidad}}
\textbf{Definición:} Se refiere al conjunto de actividades y responsabilidades llevadas a cabo por una entidad o profesional designado para gestionar eficientemente una comunidad de copropietarios. 
\subsection{\textcolor{teal}{Comercial}:{Agencia de marketing}}
\textbf{Definición:} Es una empresa encargada de gestionar las actividades de marketing, con el objetivo de atraer consumidores, incrementar las ventas, fortalecer el branding y generar valor para el mercado.
\subsection{\textcolor{teal}{Comercial}:{Always on (AO)}}
\textbf{Definición:} Hace referencia a las campañas de marketing que están activas de forma constante para responder a la demanda de un producto o servicio.
\subsection{\textcolor{teal}{Comercial}:{Apertura de cuenta corriente y obtención de rol de la comunidad}}
\textbf{Definición:} Procedimiento para permitir que la comunidad opere con fondos, ya sea a través de cheques o transferencias. Incluye la apertura de cuentas corrientes, solicitud de documentación necesaria y la asignación de roles y responsabilidades de la comunidad ante el Servicio de Impuestos Internos (SII).
\subsection{\textcolor{teal}{Comercial}:{Asamblea de Copropietarios}}
\textbf{Definición:} Reunión periódica de los propietarios de la comunidad, donde se discuten temas importantes, se toman decisiones relevantes y se abordan asuntos clave relacionados con la gestión y el desarrollo de la comunidad
\subsection{\textcolor{teal}{Comercial}:{Asignación de roles SCI}}
\textbf{Definición:} Corresponde a la asignación de roles en el sistema SCI de acuerdo al perfil del trabajador.
\subsection{\textcolor{teal}{Comercial}:{Backlink (retroenlace)}}
\textbf{Definición:} Corresponde a un enlace hecho de un sitio web a otro. Es un enlace colocado estratégicamente en el texto para guiar al lector a la página de tu blog o sitio web donde lo esperan contenidos relevantes. Su objetivo es ampliar la cantidad de navegantes que te conocen y, al mismo tiempo, generar tráfico de calidad hacia tu blog o sitio web. 
\subsection{\textcolor{teal}{Comercial}:{Calidad experiencia en la cotización/compra sala de ventas (cliente incógnito)}}
\textbf{Definición:} Evaluación de la calidad de la experiencia de cotización y compra de una vivienda a través de la visita a la sala de ventas por parte de un cliente incógnito o "mystery shopper". Se analizan aspectos como la atención del ejecutivo, la presentación de la oferta, la información proporcionada y la calidad del proceso de cotización y compra.
\subsection{\textcolor{teal}{Comercial}:{Cambio de administración}}
\textbf{Definición:} Proceso que implica la transición de una administración a otra, ya sea por elección de la comunidad o por término del contrato con la administración actual. Incluye la transferencia de documentos, registros y responsabilidades.
\subsection{\textcolor{teal}{Comercial}:{Campaña de marketing}}
\textbf{Definición:} Corresponden a estrategias planificadas y coordinadas que se llevan a cabo para promocionar un proyecto o una marca específica. Estas campañas pueden incluir una combinación de diferentes tácticas de marketing para alcanzar los objetivos establecidos, como publicidad en medios digitales, redes sociales, relaciones públicas, eventos, entre otros.
\subsection{\textcolor{teal}{Comercial}:{Cliente incógnito (mystery shopper)}}
\textbf{Definición:} Un mystery shopper, comprador misterioso o cliente incógnito es una persona anónima, encargada y enviada por una empresa para comprobar la calidad y el buen funcionamiento de sus servicios comerciales (vendedores, asesores, etc.) o de información (servicio postventa, por ejemplo).
\subsection{\textcolor{teal}{Comercial}:{Cliente referido}}
\textbf{Definición:} Corresponde al cliente que ha comprado una unidad motivado por la recomendación de otro cliente.
\subsection{\textcolor{teal}{Comercial}:{Clusterización}}
\textbf{Definición:} Es la categorización de la información del consumidor para generar segmentaciones relevantes para la campaña. Son grupos de personas con características similares.
\subsection{\textcolor{teal}{Comercial}:{Cobros anticipados o prorrogas de pagos al cliente}}
\textbf{Definición:} Corresponde a la posibilidad de que un cliente pague parte o la totalidad del precio de un inmueble antes de la entrega del mismo, o de que se acuerde una prórroga en el plazo de pago de alguna cuota establecida en el contrato de venta. 
\subsection{\textcolor{teal}{Comercial}:{Comisión vendedores}}
\textbf{Definición:} Corresponde a la cantidad de UF a pago por año de cada vendedor. Para ello se calculan las unidades vendidas (metros cuadrados) de cada vendedor.
\subsection{\textcolor{teal}{Comercial}:{Compraventa, Escritura de Compraventa o Borrador de Escritura}}
\textbf{Definición:} Documento de vital importancia a la hora de adquirir un bien raíz, ya que establece jurídicamente las obligaciones y los derechos de la persona que adquiere una vivienda o un terreno.
\subsection{\textcolor{teal}{Comercial}:{Comunicaciones con el cliente - Posventa}}
\textbf{Definición:} Registros detallados de todas las interacciones y conversaciones mantenidas con el cliente después de la compra de la propiedad. Esto incluye llamadas telefónicas, correos electrónicos, reuniones y correspondencia.
\subsection{\textcolor{teal}{Comercial}:{Comunidad de copropietarios}}
\textbf{Definición:} Se refiere a un conjunto de propiedades inmobiliarias, en los cuales los residentes son propietarios individuales de unidades específicas. Esta comunidad comparte áreas y servicios comunes, como pasillos, áreas verdes, y servicios generales del edificio. La gestión y operación de la comunidad se realizan mediante la colaboración de los copropietarios y una administración designada.
\subsection{\textcolor{teal}{Comercial}:{Comunity manager}}
\textbf{Definición:}  Persona responsable de construir y administrar la comunidad online y gestionar la identidad y la imagen de marca, creando y manteniendo relaciones estables y duraderas con sus clientes.
\subsection{\textcolor{teal}{Comercial}:{Configuración comercial SCI}}
\textbf{Definición:} Rol de administrador de producto. Corresponde al nexo entre ventas y soporte, y es responsable de configuraciones y modificaciones de los proyectos inmobiliarios en el sistema SCI.
\subsection{\textcolor{teal}{Comercial}:{Costo por adquisición (CPA)}}
\textbf{Definición:} Corresponde a costo total que se invierte en publicidad online para lograr que un usuario realice una compra. El anunciante solo paga cuando la acción deseada por el anunciante se lleva a cabo por los visitantes. Por tanto, es de bajo riesgo para quienes anuncian, porque el pago se realiza sólo por las transacciones de los que realizaron lo que deseabas.
\subsection{\textcolor{teal}{Comercial}:{Creación de proyecto en SCI}}
\textbf{Definición:} Corresponde a la creación de un proyecto comercial en el sistema SCI, es decir, el ingreso del proyecto inmobiliario y sus características, como lo son, el número de unidades y tipologías, entre otros.
\subsection{\textcolor{teal}{Comercial}:{Crédito Directo Socovesa (CDS)}}
\textbf{Definición:} Corresponde a una forma de financiamiento que entregaba la empresa a sus clientes en la compra de una propiedad. Actualmente esta modalidad de financiamiento no está vigente. Sin embargo, quedan algunas cuotas pendientes de pago.
\subsection{\textcolor{teal}{Comercial}:{Datos de análisis de campañas de Google Ads}}
\textbf{Definición:} Son los datos obtenidos y analizados a partir de las campañas de publicidad que se ejecutan a través de la plataforma de Google Ads. Estos datos pueden incluir métricas como clics, impresiones, CTR (Click-Through Rate), conversiones, costos por clic (CPC), tasas de conversión y otros indicadores que ayudan a evaluar el rendimiento y el éxito de las campañas de publicidad en Google.
\subsection{\textcolor{teal}{Comercial}:{Datos de análisis de redes sociales}}
\textbf{Definición:} Son los datos recopilados y analizados a partir de las actividades y el comportamiento de los usuarios en diferentes plataformas de redes sociales. Estos datos pueden incluir métricas como el número de seguidores, alcance, interacciones, menciones, menciones de marca, comentarios, compartidos y otros datos relevantes para evaluar el rendimiento y la efectividad de las estrategias de marketing en redes sociales.
\subsection{\textcolor{teal}{Comercial}:{Datos de análisis de SEO}}
\textbf{Definición:} Son los datos utilizados para evaluar y medir el rendimiento de las estrategias de optimización para motores de búsqueda (SEO). Estos datos pueden incluir métricas como el ranking en los resultados de búsqueda, el tráfico orgánico, las palabras clave más efectivas, los enlaces entrantes, la tasa de rebote, la duración de la visita y otros indicadores relevantes para mejorar la visibilidad y el posicionamiento en los motores de búsqueda.
\subsection{\textcolor{teal}{Comercial}:{Datos de contacto del cotizante}}
\textbf{Definición:} Corresponde al grupo de datos que permite establecer una comunicación con el cotizante. Estos corresponden a: Dirección, comuna, correo electrónico y teléfono. Esta información es obtenida a través de portales de información (Enlace inmobiliario, Portal inmobiliario, TOC-TOC y facebook leads), sitios web de las marcas y a través de cotización en el sistema SCI.
\subsection{\textcolor{teal}{Comercial}:{Datos de cotizantes (Enlace inmobiliario)}}
\textbf{Definición:} Corresponde al conjunto de datos que identifican y permiten establecer una comunicación con el cotizante, y es obtenida a través del portal inmobiliario "Enlace inmobiliario". Estos datos son los siguientes: nombre cliente, apellido cliente, RUT, correo electrónico y teléfono.
\subsection{\textcolor{teal}{Comercial}:{Datos de cotizantes (leads de Meta)}}
\textbf{Definición:} Corresponde al conjunto de datos que identifican y permiten establecer una comunicación con el cotizante, y es obtenida a través del portal inmobiliario "Enlace inmobiliario". Estos datos son los siguientes: nombre cliente, apellido cliente, RUT, correo electrónico y teléfono.

Corresponde a la información recopilada de cotizantes que han expresado interés en algún proyecto a través de la plataforma Meta. Estos datos generalmente incluyen información personal y de contacto proporcionada por los usuarios que han completado un formulario o dejado sus datos en el sitio web, aplicación móvil u otros canales de captura de leads de Meta. Estos datos corresponden a nombre y apellido, RUT, correo electrónico, teléfono, preferencia de compra (habitar o inversión).
\subsection{\textcolor{teal}{Comercial}:{Datos de cotizantes (sitio web)}}
\textbf{Definición:} Corresponde al conjunto de datos que identifican y permiten establecer una comunicación con el cotizante, y es obtenida a través del sitio web de la marca. Estos datos son los siguientes: nombre completo, correo electrónico, RUT y teléfono.
\subsection{\textcolor{teal}{Comercial}:{Datos de identificación del cotizante}}
\textbf{Definición:} Corresponde al conjunto de datos que se relacionan con el cotizante y los identifican. Estos datos son los siguientes: nombre, apellido, RUT, género, estado civil y ocupación. Esta información es obtenida a través de portales de información (Enlace inmobiliario, Portal inmobiliario, TOC-TOC y facebook leads), sitios web de marcas y a través de cotización en el sistema SCI.
\subsection{\textcolor{teal}{Comercial}:{Datos del cotizantes (Portal Inmobiliario)}}
\textbf{Definición:} Corresponde al conjunto de datos que identifican y permiten establecer una comunicación con el cotizante, y es obtenida a través del portal inmobiliario "Portal inmobiliario". Estos datos son los siguientes: nombre y apellido cliente, RUT, correo electrónico y teléfono.
\subsection{\textcolor{teal}{Comercial}:{Datos del referido (sitio web)}}
\textbf{Definición:} Corresponde al conjunto de datos que identifican y permiten establecer una comunicación con el cotizante referido por otro cliente, y es obtenida a través del sitio web de la marca. Estos datos son los siguientes: nombre, apellido, correo electrónico, RUT y teléfono.
\subsection{\textcolor{teal}{Comercial}:{Desistimiento reserva de inmueble}}
\textbf{Definición:} Corresponde a la facultad de un cliente de dejar si efecto la reserva del inmueble, notificándoselo así a la otra parte
\subsection{\textcolor{teal}{Comercial}:{Elección de la primera administración}}
\textbf{Definición:} El proceso de selección y designación de la primera administración para una comunidad. Esto implica evaluar propuestas, seleccionar al administrador adecuado y llevar a cabo los trámites necesarios para formalizar la relación con la administración elegida.
\subsection{\textcolor{teal}{Comercial}:{Encuesta calidad de servicio de la entrega}}
\textbf{Definición:} Encuesta elaborada por el área de Research, cuyo objetivo es evaluar la experiencia de los clientes durante la entrega de la vivienda. Los datos recopilados incluyen experiencia de calidad de servicio durante la entrega de la vivienda (1 mes después de la entrega), estado de la vivienda entregada y nivel de recomendación (NPF) durante la entrega.
\subsection{\textcolor{teal}{Comercial}:{Encuesta calidad de servicio de la espera}}
\textbf{Definición:} Encuesta elaborada por el área de Research, cuyo objetivo es evaluar la experiencia de los clientes durante el periodo de espera previo a la entrega de la vivienda. Los datos recopilados incluyen experiencia de calidad de servicio durante la espera de los clientes (4 meses después de la firma), atención del ejecutivo durante la espera y nivel de recomendación (NPS) durante la espera.
\subsection{\textcolor{teal}{Comercial}:{Encuesta de calidad de servicio de la posventa}}
\textbf{Definición:} Encuesta elaborada por el área de Research, cuyo objetivo es evaluar la experiencia de los clientes en el periodo posterior a la entrega de la vivienda. Los datos recopilados incluyen experiencia de calidad de servicio durante la posventa de la vivienda (2 meses después de solicitar el servicio de posventa), atención posventa recibida (o no recibida) y nivel de recomendación (NPF) durante la posventa.
\subsection{\textcolor{teal}{Comercial}:{Encuesta de evaluación de campañas de marcas}}
\textbf{Definición:} Encuesta realizada por el área de Research, cuyo objetivo es evaluar la efectividad y percepción de las campañas de marca realizadas por la empresa. La información obtenida corresponde a la evaluación de la campaña en términos de impacto y relevancia, imagen proyectada por la campaña, nivel de alcance de la campaña y la retención en la mente del público objetivo.
\subsection{\textcolor{teal}{Comercial}:{Encuesta desempeño de producto}}
\textbf{Definición:} Encuesta elaborada por el área de Research, cuyo objetivo es evaluar el nivel de satisfacción de los clientes con el producto (vivienda) y obtener información sobre su uso. Los datos recopilados incluyen satisfacción del cliente con la vivienda, modificaciones realizadas a la vivienda por parte del cliente y lugares más utilizados dentro de la vivienda.
\subsection{\textcolor{teal}{Comercial}:{Encuesta Perfil de cotizantes y preferencia de clientes}}
\textbf{Definición:} Encuesta elaborada por el área de Research, cuyo objetivo es recopilar información sobre los cotizantes (propios y del mercado) y las preferencias de los clientes en relación a los proyectos de Socovesa. Algunos de los datos que recopila corresponden a datos demográficos, características de los proyectos inmobiliarios que más les interesan, ubicación preferida, tipo de vivienda, amenidades, etc.; preferencias generales de los potenciales compradores en el mercado inmobiliario; y preferencias y necesidades específicas de clientes Socovesa.
\subsection{\textcolor{teal}{Comercial}:{Encuesta Radiografía del Comprador}}
\textbf{Definición:} Encuesta elaborada por el área de Research, cuyo objetivo es recopilar información sobre el perfil de los compradores de proyectos Socovesa (clientes). Algunos de los datos que recopila corresponden a características demográficas, preferencias, comportamientos de compra y necesidades de los compradores.
\subsection{\textcolor{teal}{Comercial}:{Entrega de la comunidad a los propietarios}}
\textbf{Definición:} El proceso formal de transferir la responsabilidad y control de la comunidad a sus propietarios. Esto puede ocurrir después de ciertos hitos, como la venta de un porcentaje mayoritario de las propiedades (aprox. 70%) o al finalizar el período de administración.
\subsection{\textcolor{teal}{Comercial}:{Estatus de entrega de la propiedad}}
\textbf{Definición:} Indica el estado actual de la entrega de la unidad, siendo "Entregado" cuando no hay observaciones, "Recibe Conforme" cuando el cliente está satisfecho, o "No Recibe Conforme" cuando el cliente rechaza la entrega debido a observaciones pendientes.
\subsection{\textcolor{teal}{Comercial}:{Estudio de cliente incógnito}}
\textbf{Definición:} Estudio realizado por el área de Research, cuyo objetivo es saber si en las salas de ventas se ofrece un servicio de calidad. El proceso consiste en que el cliente incógnito (agencia) finge ser un cotizante común que visita la sala de ventas para evaluar a los ejecutivos de venta en la sala de venta y las condiciones físicas de dicha sala.
\subsection{\textcolor{teal}{Comercial}:{Estudio de no compra}}
\textbf{Definición:} Encuesta elaborada por el área de Research, cuyo objetivo es recopilar información sobre los motivos por los cuales los cotizantes no realizaron la compra de una vivienda. Los datos recopilados incluyen motivos de no compra de los cotizantes, evaluación de la sala de venta, evaluación de la atención del ejecutivo, evaluación del piloto físico, evaluación del piloto virtual, evaluación del sitio web y evaluación del proyecto en comparación con la competencia.
\subsection{\textcolor{teal}{Comercial}:{Estudio de valor corporativo}}
\textbf{Definición:} Encuesta elaborada por el área de Research, cuyo objetivo es evaluar la experiencia de los cotizantes en la sala de venta y la atención recibida. Los datos recopilados incluyen la evaluación de los siguientes factores: 
piloto (incluye piloto virtual), experiencia de los clientes respecto a la sala de venta, ejecutivos de venta, atención recibida, calidad de la videollamada y calidad del sitio web.
\subsection{\textcolor{teal}{Comercial}:{Estudio de valor de proyectos}}
\textbf{Definición:} Encuesta elaborada por el área de Research, cuyo objetivo evaluar internamente el valor de los proyectos inmobiliarios y obtener la opinión de los cotizantes sobre el valor de dichos proyectos, para determinada marca. Los datos recopilados incluyen evaluación interna del valor de los proyectos inmobiliarios y opiniones de los cotizantes propios sobre el valor de los proyectos.
\subsection{\textcolor{teal}{Comercial}:{Estudio e imagen de marca}}
\textbf{Definición:} Encuesta realizada por el área de Research, cuyo objetivo es evaluar la imagen de marca de la empresa y obtener información sobre las preferencias de los clientes en relación a la marca. Algunos de los datos recopilados por esta encuesta son: atributos asociados a la marca, imagen de marca percibida por los clientes, marcas de la competencia, participación de mercado cualitativa y evaluación de la marca en diferentes aspectos.
\subsection{\textcolor{teal}{Comercial}:{Estudio etnográfico (entrevista insight)}}
\textbf{Definición:} Entrevista elaborada por el área de Research, cuyo objetivo es obtener información detallada sobre los cotizantes y compradores mediante entrevistas en profundidad. Los datos recopilados corresponden a caracterización de la composición familiar, estilo de vida, situación actual, motivación de búsqueda, exploración del departamento ideal, evaluación del proyecto inmobiliario, evaluación de elementos específicos (detalles, terminaciones o espacios) del proyecto inmobiliario por parte de los cotizantes y compradores; y recomendaciones para el departamento ideal.
\subsection{\textcolor{teal}{Comercial}:{Estudio etnográfico (entrevista insitu)}}
\textbf{Definición:} Entrevista elaborada por el área de Research, cuyo objetivo es obtener información sobre los residentes actuales de los proyectos inmobiliarios de Socovesa mediante entrevistas en su lugar de residencia. Los datos recopilados corresponden a caracterización de la composición familiar, estilo de vida, características de la vivienda anterior, propuesta de valor, motivaciones de compra, rutinas cotidianas y cambios de rutina post pandemia.
\subsection{\textcolor{teal}{Comercial}:{Evaluación del cliente - atributos, imagen y evaluación de marca}}
\textbf{Definición:} Proceso de evaluación de la percepción y valoración de los clientes en relación a los atributos y la imagen de marca. Se analizan aspectos como la identidad de marca, la reputación, la confianza y la conexión emocional con la marca.
\subsection{\textcolor{teal}{Comercial}:{Evaluación del cliente - venta de valor corporativo}}
\textbf{Definición:} Evaluación de la experiencia y satisfacción del cliente durante el proceso de venta, enfocado en el valor corporativo proporcionado por la empresa. Se analizan aspectos como la experiencia en la sala de venta, la atención recibida, la calidad de la comunicación y la capacidad de la empresa para transmitir y cumplir las propuestas de valor.
\subsection{\textcolor{teal}{Comercial}:{Evaluación del cliente - venta de valor proyecto}}
\textbf{Definición:} Evaluación de la percepción y valoración de los clientes en relación al valor ofrecido por un proyecto inmobiliario específico. Se analizan aspectos como la propuesta de valor, las características y amenidades del proyecto, la relación calidad-precio y la satisfacción general con el proyecto.
\subsection{\textcolor{teal}{Comercial}:{Evaluación en público objetivo de la campaña, imagen proyectada, nivel de alcance y retención}}
\textbf{Definición:} Evaluación de una campaña publicitaria por parte del público objetivo. Se analiza la imagen proyectada por la campaña, el nivel de alcance en el público objetivo y la capacidad de la campaña para retener la atención y generar impacto en la audiencia.
\subsection{\textcolor{teal}{Comercial}:{Experiencia clientes calidad de servicio en la entrega de la vivienda}}
\textbf{Definición:} Evaluación de la experiencia de los clientes durante el proceso de entrega de la vivienda. Se analizan aspectos como la calidad de la vivienda entregada, la atención recibida y la satisfacción general con el servicio durante esta etapa.
\subsection{\textcolor{teal}{Comercial}:{Experiencia clientes calidad de servicio en la espera de la vivienda}}
\textbf{Definición:} Evaluación de la experiencia de los clientes durante el periodo de espera previo a la entrega de la vivienda. Se analizan aspectos como la atención del ejecutivo, la comunicación y la calidad del servicio durante esta etapa.
\subsection{\textcolor{teal}{Comercial}:{Experiencia clientes calidad de servicio en la posventa de la vivienda}}
\textbf{Definición:} Evaluación de la experiencia de los clientes en el periodo posterior a la entrega de la vivienda. Se analizan aspectos como la atención posventa, el soporte técnico, la resolución de problemas y la satisfacción general con el servicio durante esta etapa.
\subsection{\textcolor{teal}{Comercial}:{Fallas en la revisión calidad/obra de la unidad}}
\textbf{Definición:} Son observaciones o problemas relacionados con la calidad del departamento, como problemas en las instalaciones, terminaciones, o aspectos como guardapolvo, cubierta, cerámica en los muros, etc. Estas fallas se registran en un checklist de revisión.
\subsection{\textcolor{teal}{Comercial}:{Fecha estimada de entrega al cliente}}
\textbf{Definición:} Fecha prevista para la entrega de la unidad inmobiliaria al cliente.
\subsection{\textcolor{teal}{Comercial}:{Feedback del Cliente - Posventa}}
\textbf{Definición:} Manifestaciones de insatisfacción, descontento o sugerencias presentadas por el cliente en relación con algún aspecto de la propiedad o del servicio posventa. Estos comentarios pueden abordar problemas específicos o expresar preocupaciones generales.
\subsection{\textcolor{teal}{Comercial}:{Flujo de cotizantes}}
\textbf{Definición:} Corresponde a la cantidad de cotizantes que tienes en una cantidad de tiempo.
\subsection{\textcolor{teal}{Comercial}:{Folleto (Brochure)}}
\textbf{Definición:} Es una herramienta de marketing que hace referencia a la documentación impresa o digital que tiene como objetivo representar a una determinada compañía para informar sobre su organización, productos o servicios. En el caso de Empresas Socovesa, para representar un proyecto inmobiliario determinado.
\subsection{\textcolor{teal}{Comercial}:{Fondos de reserva comunidad}}
\textbf{Definición:} Fondos necesarios para la operación y mantenimiento de la comunidad, que varían según el número de unidades en el edificio. Estos fondos pueden ser utilizados para gastos comunes, compra de suministros y mobiliario, y son gestionados por la administración en colaboración con la inmobiliaria. La devolución de estos fondos a la inmobiliaria es responsabilidad de la administración.
\subsection{\textcolor{teal}{Comercial}:{Funnel de ventas}}
\textbf{Definición:} También conocido como el embudo de conversión o el embudo de ventas, es un término de Marketing Digital que engloba el proceso y los distintos pasos que un contacto digital da dentro de la web hasta cumplir un objetivo, en este caso, concretar una venta.
\subsection{\textcolor{teal}{Comercial}:{Garantías de posventa}}
\textbf{Definición:} Compromisos contractuales que aseguran que ciertos aspectos de la propiedad o los servicios posventa cumplirán con estándares predefinidos. En caso de defectos dentro de un período determinado, la garantía garantiza la reparación o el reemplazo correspondiente.
\subsection{\textcolor{teal}{Comercial}:{Google Addwords}}
\textbf{Definición:} Corresponde a un servicio de publicidad que ofrece Google, el cual consiste en comprar un listado de palabras que tengan relación con el negocio, de modo que cada vez que éstas sean buscadas por los usuarios en los buscadores web, aparezca el sitio web de la empresa en forma de anuncio al inicio de las listas de resultados. Este servicio de Google sirve para generar mayor tráfico al sitio web, captar nuevos clientes, así como obtener mayor alcance.
\subsection{\textcolor{teal}{Comercial}:{Indicadores de campañas de marketing}}
\textbf{Definición:} Son métricas o medidas utilizadas para evaluar el desempeño y los resultados de las campañas de marketing.
\subsection{\textcolor{teal}{Comercial}:{Información del cliente Posventa}}
\textbf{Definición:} Información detallada sobre el cliente que ha adquirido una propiedad, incluyendo su nombre, datos de contacto, detalles específicos de la propiedad adquirida y cualquier otra información relevante para la gestión posventa.
\subsection{\textcolor{teal}{Comercial}:{Información etnográfica de cotizantes y compradores (inside)}}
\textbf{Definición:} Datos obtenidos a través de entrevistas a cotizantes y compradores en los cuales se investiga y se recopila información acerca de la composición familiar, estilo de vida, motivaciones, evaluaciones y recomendaciones relacionadas con la compra de una vivienda.
\subsection{\textcolor{teal}{Comercial}:{Información etnográfica de residentes (insitu)}}
\textbf{Definición:} Datos obtenidos a través de entrevistas a residentes de los proyectos, en los cuales se investiga y se recopila información sobre la composición familiar, estilo de vida, rutinas cotidianas, evaluaciones y cambios de rutina relacionados con la vida en la vivienda o proyecto.
\subsection{\textcolor{teal}{Comercial}:{Intención de búsqueda (keywords - kws)}}
\textbf{Definición:} La intención de búsqueda se refiere a lo que el usuario espera encontrar al realizar una búsqueda en los motores de búsqueda, según la consulta enviada. A medida que la consulta del usuario se vuelve más específica, se esperaría obtener resultados más relevantes. Sin embargo, incluso una sola palabra puede influir en las expectativas del usuario y cambiar lo que espera ver.
\subsection{\textcolor{teal}{Comercial}:{Journey Builder}}
\textbf{Definición:} Herramienta de salesforce que permite un flujo de comunicación automatizada o no automatizada con el potencial cliente.
\subsection{\textcolor{teal}{Comercial}:{Leads}}
\textbf{Definición:} Corresponde a un cliente potencial que demostró interés en alguna unidad. Los leads provienen de la cotización web o de las herramientas de "Facebook leads". 
\subsection{\textcolor{teal}{Comercial}:{Linkbuilding (construcción de enlaces)}}
\textbf{Definición:} El link building es una estrategia SEO que consiste en construir una red de enlaces internos y backlinks (retroenlaces) con la que se aumente el rendimiento de una página web mediante referencias de calidad y notoriedad en otras páginas. El objetivo de esta técnica es crear una serie de enlaces externos para generar autoridad web y mejorar el posicionamiento orgánico en buscadores.
\subsection{\textcolor{teal}{Comercial}:{Llamado de coordinación}}
\textbf{Definición:} Comunicación para coordinar acciones o resolver aspectos específicos relacionados con la entrega.
\subsection{\textcolor{teal}{Comercial}:{Marketing orgánico}}
\textbf{Definición:} Es una estrategia que genera tráfico a la empresa a lo largo del tiempo en lugar de utilizar métodos de pago. Esto incluye publicaciones de blogs, estudios de casos, publicaciones de invitados, tuits no pagados y actualizaciones de Facebook. El marketing orgánico utiliza el SEO, redes sociales y una variedad de otros canales para aumentar el conocimiento de la marca.
\subsection{\textcolor{teal}{Comercial}:{Mensajes de Mailing masivo}}
\textbf{Definición:} Se refiere al envío de mensajes de correo electrónico a cotizantes/clientes de forma simultánea. Estos mensajes suelen utilizarse para difundir información, promociones, noticias o actualizaciones relevantes sobre algún proyecto a una base de datos de contactos.
\subsection{\textcolor{teal}{Comercial}:{Mensajes de SMS masivo}}
\textbf{Definición:} Consiste en el envío de mensajes de texto cortos a clientes/cotizantes al mismo tiempo. Los mensajes de SMS masivos son una forma rápida y directa de comunicación que se utiliza para enviar promociones, recordatorios, alertas u otra información relevante a los usuarios.
\subsection{\textcolor{teal}{Comercial}:{Mensajes de Whatsapp masivo}}
\textbf{Definición:} Implica el envío de mensajes a través de la plataforma de mensajería instantánea WhatsApp a clientes/cotizantes simultáneamente. Esta táctica se utiliza para enviar comunicaciones promocionales, actualizaciones de productos, atención al cliente u otro tipo de mensajes relevantes.
\subsection{\textcolor{teal}{Comercial}:{Modificaciones promesa de compraventa}}
\textbf{Definición:} Corresponden a las modificaciones que se realizan en la promesa de compraventa, ya sea por modificaciones a la unidad adquirida, cambio de unidades secundarias, cambia de oferente, cambio en los vencimientos, cambio en la forma de pago, entre otras.
\subsection{\textcolor{teal}{Comercial}:{Motivos de no compra}}
\textbf{Definición:} Razones por las cuales los cotizantes o potenciales clientes deciden no realizar la compra de una vivienda o proyecto inmobiliario. Estos motivos pueden incluir factores como precio, ubicación, características del proyecto, insatisfacción con la oferta o preferencia por otras alternativas en el mercado.
\subsection{\textcolor{teal}{Comercial}:{Nivel de recomendación o Net Promoter Score (NPS)}}
\textbf{Definición:} El NPS es una métrica utilizada para medir la lealtad y satisfacción de los clientes con una empresa, producto o servicio en particular. Se basa en una pregunta simple: "En una escala del 0 al 10, ¿qué tan probable es que recomiendes nuestra empresa/producto/servicio a un amigo o colega?".
\subsection{\textcolor{teal}{Comercial}:{Nota revisión de calidad a la constructora}}
\textbf{Definición:} Nota de 1.0 a 10 que registra el resultado de la revisión de calidad realizada en el departamento y que depende del tipo de observaciones (criticidad, leve, grave, muy grave)
\subsection{\textcolor{teal}{Comercial}:{Nutrición de leads (Lead nurturing)}}
\textbf{Definición:} Proceso deliberado de atraer a un grupo objetivo ofreciéndole información relevante en cada una de las fases del recorrido del comprador, con el fin de posicionar a la empresa como la apuesta más segura para ayudarlos a alcanzar sus objetivos.
\subsection{\textcolor{teal}{Comercial}:{Observaciones durante la entrega al cliente}}
\textbf{Definición:} Comentarios y evaluaciones sobre la calidad del departamento realizadas por los clientes después de recibir la unidad, proporcionando una retroalimentación valiosa para mejorar los estándares de entrega.
\subsection{\textcolor{teal}{Comercial}:{Órdenes de compra de marketing}}
\textbf{Definición:} Son solicitudes para la adquisición de bienes o servicios relacionados con actividades de marketing. Estas órdenes de compra se generan para adquirir recursos como impresiones publicitarias, servicios de agencias de marketing, software especializado, materiales promocionales, entre otros.
\subsection{\textcolor{teal}{Comercial}:{Ordenes de trabajo para reparaciones y mantenimientos - Posventa}}
\textbf{Definición:} Registros detallados que describen las tareas específicas a realizar en una propiedad como parte de trabajos de reparación o mantenimiento. Incluyen información sobre la naturaleza de las labores, los costos asociados y las fechas programadas para la ejecución.
\subsection{\textcolor{teal}{Comercial}:{Pedido de materiales - Posventa}}
\textbf{Definición:} Solicitud formal realizada por los jefes de terreno para obtener los insumos necesarios destinados a actividades de posventa en unidades inmobiliarias. 
\subsection{\textcolor{teal}{Comercial}:{Perfil de cliente}}
\textbf{Definición:} Datos obtenidos a través de encuestas a clientes. Corresponde a las características demográficas, preferencias, comportamientos y necesidades de los clientes de la empresa, con el objeto de saber quiénes son y qué los motiva a adquirir proyectos Socovesa.
\subsection{\textcolor{teal}{Comercial}:{Perfilamiento de clientes}}
\textbf{Definición:} Corresponde a un resumen de las características del consumidor ideal de una empresa. Se basa en información estadística (sexo, edad, ingresos), psicográfica (estilo de vida, valores, deseos) y conductual (hábitos y frecuencia de compra).
\subsection{\textcolor{teal}{Comercial}:{Piezas de marketing digital}}
\textbf{Definición:} Son elementos creativos y visuales utilizados en el ámbito del marketing digital para promover productos, servicios o marcas. Estas piezas pueden incluir imágenes, banners, videos, infografías, GIF, audios u otros formatos digitales diseñados para atraer y captar la atención del público objetivo.
\subsection{\textcolor{teal}{Comercial}:{Planilla de entregas}}
\textbf{Definición:} Reporte del Sistema de Revisión de Departamentos (SRD) que recopila información detallada de todas las entregas programadas, registrando su estado (rechazada o aprobada), la fecha de revisión, la aprobación y el revisor, así como el número de observaciones identificadas.
\subsection{\textcolor{teal}{Comercial}:{Póliza de seguro venta en verde}}
\textbf{Definición:} Corresponde al seguro de garantía de venta en verde que permite a compradores de inmuebles en construcción tener seguridad sobre los dineros anticipados en caso de que la inmobiliaria no cumpla con sus obligaciones.
\subsection{\textcolor{teal}{Comercial}:{Portal de información inmobiliaria}}
\textbf{Definición:} Corresponden a sitios web especializados en la publicación y búsqueda de propiedades inmobiliarias, ya sean de venta o alquiler. Algunos ejemplos son: portal inmobiliario, toc-toc y enlace inmobiliario.
\subsection{\textcolor{teal}{Comercial}:{Posventa}}
\textbf{Definición:} La supervisión continua de las actividades y pendientes de la comunidad durante un período de hasta 5 años después de la compra de la propiedad.
\subsection{\textcolor{teal}{Comercial}:{Pre reserva de inmueble}}
\textbf{Definición:} Estado transitorio de venta en la cual el cliente ha manifestado la intención de reserva pero no ha hecho el pago respectivo.
\subsection{\textcolor{teal}{Comercial}:{Preferencias de los cotizantes (propios y del mercado)}}
\textbf{Definición:} Datos obtenidos a través de encuestas a cotizantes propios y del mercado. Corresponde a las características de los proyectos inmobiliarios que son de interés, ubicación preferida, tipo de vivienda, amenidades y otras preferencias generales respecto de un proyecto habitacional.
\subsection{\textcolor{teal}{Comercial}:{Proceso de licitación de administraciones}}
\textbf{Definición:} El procedimiento para seleccionar la primera administración de un proyecto. Este proceso implica la evaluación de propuestas y la elección de una entidad administrativa que cumpla con los requisitos y expectativas establecidos.
\subsection{\textcolor{teal}{Comercial}:{Promesa de Compraventa}}
\textbf{Definición:} contrato que se hace entre un comprador y un vendedor de una propiedad antes de firmar la escritura de compraventa.
\subsection{\textcolor{teal}{Comercial}:{Promociones y acuerdos comerciales al cliente }}
\textbf{Definición:} Corresponde a incentivos ofrecidos al cliente como parte de la negociación en la venta de la propiedad, tales como, el pago por referidos o la entrega de una giftcard, entre otros.
\subsection{\textcolor{teal}{Comercial}:{Propiedad en parte de pago}}
\textbf{Definición:} propiedad usada que se entrega a la Inmobiliaria cuyo diferencial entre el precio y la deuda (si la hubiera) formará parte del pie de la compra de la propiedad nueva.
\subsection{\textcolor{teal}{Comercial}:{Proyectos con unidades por entregar}}
\textbf{Definición:} Se refiere a proyectos habitacionales, como departamentos o casas, que aún tienen unidades disponibles para ser entregadas a los clientes. Estas unidades pueden estar escrituradas (legalmente transferidas) o no, y unidades que no se han vendido (disponible)
\subsection{\textcolor{teal}{Comercial}:{Recaudación de saldos reales por diferencia de UF}}
\textbf{Definición:} Se refiere al proceso de ajuste de los pagos realizados por un cliente, en relación a la diferencia entre el valor de la Unidad de Fomento (UF) al momento en que se realizó la compra de la propiedad y el valor actual de la UF al momento del pago.
\subsection{\textcolor{teal}{Comercial}:{Recuperación de subsidios y ahorro}}
\textbf{Definición:} La recuperación de subsidios y ahorro se refiere al proceso mediante el cual una inmobiliaria recupera los subsidios y ahorros utilizados por los clientes para la compra de una propiedad. 
\subsection{\textcolor{teal}{Comercial}:{Remarketing}}
\textbf{Definición:} Hacer marketing nuevamente para la misma persona. La intención es generar impacto más de una vez sobre alguien que ya demostró interés en el producto.
\subsection{\textcolor{teal}{Comercial}:{Render}}
\textbf{Definición:} Se traduce como retratar o representar y corresponde a una imagen digital que se crea a partir de un modelo o escenario 3D realizado en algún programa de computadora especializado, cuyo objetivo es dar una apariencia realista.
\subsection{\textcolor{teal}{Comercial}:{Resciliación de promesa compraventa}}
\textbf{Definición:} Corresponde a la extinción de las obligaciones provenientes de la promesa de compraventa, una vez convenido entre las partes involucradas.
\subsection{\textcolor{teal}{Comercial}:{Reserva de inmueble}}
\textbf{Definición:} Un contrato de reserva de vivienda es aquel a través del cual un comprador reserva la vivienda a cambio del pago de una cantidad de dinero que posteriormente se restará del precio de la misma, el vendedor se compromete a entregar el inmueble. Con la reserva, el comprador demuestra a la inmobiliaria su interés y compromiso de comprar la vivienda.
\subsection{\textcolor{teal}{Comercial}:{Retargeting}}
\textbf{Definición:} Técnica de marketing digital cuyo objetivo es impactar a los usuarios que previamente han interactuado con una determinada marca. La finalidad del retargeting es recordar a los usuarios interesados en nuestros productos que estamos ahí y que tenemos una oferta interesante que ofrecerles
\subsection{\textcolor{teal}{Comercial}:{Revisiones de calidad/ obra de la unidad}}
\textbf{Definición:} Es la iteración y revisión del departamento entre la constructora (responsable de la obra) y la inmobiliaria (responsable de revisar la unidad para que esté en condiciones de entrega a propietario).
\subsection{\textcolor{teal}{Comercial}:{Satisfacción del cliente desempeño del producto (vivienda)}}
\textbf{Definición:} Evaluación del grado de satisfacción de los clientes con el desempeño y calidad de la vivienda adquirida. Se analizan aspectos como el diseño, los acabados, el funcionamiento de las instalaciones y el cumplimiento de las expectativas del cliente.
\subsection{\textcolor{teal}{Comercial}:{Search Engine Marketing (SEM)}}
\textbf{Definición:} Conjunto de herramientas, técnicas y estrategias que nos ayudan a optimizar la visibilidad de sitios y páginas web a través de los motores de los buscadores.
\subsection{\textcolor{teal}{Comercial}:{Search Engine Optimization (SEO)}}
\textbf{Definición:} El SEO, conocido como posicionamiento en buscadores u optimización en motores de búsqueda, se refiere a un conjunto de estrategias destinadas a mejorar la posición de un sitio web en la lista de resultados de motores de búsqueda.
\subsection{\textcolor{teal}{Comercial}:{Segmentación de mercado}}
\textbf{Definición:} Corresponde a la división de los clientes en segmentos individuales homogéneos basados en características comunes. Por lo tanto, dentro de cada segmento, los clientes son similares y cada segmento contiene compradores diferentes.
\subsection{\textcolor{teal}{Comercial}:{Seguimiento de escrituración posfirma}}
\textbf{Definición:} Se refiere al proceso de gestión y revisión de los documentos necesarios para completar la escrituración de una propiedad después de la firma del contrato de compraventa.
\subsection{\textcolor{teal}{Comercial}:{Seguimiento de escrituración pre-firma}}
\textbf{Definición:} Se refiere al proceso de revisión y gestión de los documentos necesarios para llevar a cabo la escrituración de una propiedad, antes de la firma del contrato de compraventa.
\subsection{\textcolor{teal}{Comercial}:{Servicio al cliente (SAC)}}
\textbf{Definición:} Es el área que se encarga de entregar respuesta a las consultas, problemas y sugerencias que los clientes tienen acerca de la marca.
\subsection{\textcolor{teal}{Comercial}:{Sitio web}}
\textbf{Definición:} Es un espacio virtual en Internet. Se trata de un conjunto de páginas web que son accesibles desde un mismo dominio o subdominio.
\subsection{\textcolor{teal}{Comercial}:{Social adds}}
\textbf{Definición:} Publicidad en redes sociales
\subsection{\textcolor{teal}{Comercial}:{Solicitudes de posventa}}
\textbf{Definición:} Peticiones o requerimientos realizados por el cliente después de la adquisición de una propiedad. Estas solicitudes pueden abarcar desde reparaciones y mantenimiento hasta otros servicios relacionados con la propiedad.
\subsection{\textcolor{teal}{Comercial}:{Stock de material - Posventa}}
\textbf{Definición:} Se refiere a la cantidad total de insumos y materiales disponibles en el almacén de la empresa o entidad responsable de la posventa de unidades inmobiliarias. Este inventario incluye materiales destinados a cubrir pedidos de posventa, asegurando que exista una cantidad suficiente para satisfacer las demandas de las actividades programadas.
\subsection{\textcolor{teal}{Comercial}:{Tasa de apertura}}
\textbf{Definición:} Tasa porcentual de personas que abren un email, respecto de los emails enviados que se enviaron.
\subsection{\textcolor{teal}{Comercial}:{Tasa de conversión}}
\textbf{Definición:} Tasa porcentual de clientes potenciales (leads) que pasan de un estado a otro, como, por ejemplo, la tasa de conversión de cotizaciones online a cotizaciones en punto de ventas, la tasa de conversión de cotizantes en salas de ventas que pasaron a reservar una unidad y finalmente, la tasa de conversión de personas con reserva a personas que efectivamente promesaron una unidad.
\subsection{\textcolor{teal}{Comercial}:{Tasa de rebote}}
\textbf{Definición:} Tasa porcentual de las personas que entran a la web o hacen click a alguno de los anuncios y de forma inmediata se retiran (rebotan), respecto del total de personas que entran a la web o hacen click en alguno de los anuncios.
\subsection{\textcolor{teal}{Comercial}:{Tipos de falla - Árbol de fallas}}
\textbf{Definición:} Un diagrama o representación gráfica que muestra de manera jerárquica los posibles fallos o problemas que pueden surgir en una propiedad y sus componentes, lo que facilita la identificación y solución de problemas en el proceso de posventa.
\subsection{\textcolor{teal}{Comercial}:{Tráfico web}}
\textbf{Definición:} Corresponde a todo el alcance, visitas y visualizaciones del sitio web. En otras palabras, se refiere al número de accesos que un contenido recibe en Internet. El modo como las personas acceden y generan tráfico para un determinado sitio puede variar.
\subsection{\textcolor{teal}{Comercial}:{Unidades entregadas}}
\textbf{Definición:} Se considera que una unidad ha sido entregada cuando se ha programado y realizado la primera visita con el cliente, y este ha aceptado las llaves del departamento o casa. Una vez entregada, firma el acta de entrega, la unidad pasa a la etapa de posventa.
\subsection{\textcolor{teal}{Comercial}:{Visto bueno de entrega al cliente}}
\textbf{Definición:} Aprobación contable que se concede cuando la situación financiera del cliente está al corriente y cumple con los requisitos para la entrega exitosa de la unidad inmobiliaria.
\section{\textcolor{blue}{Construcción}}
\subsection{\textcolor{teal}{Construcción}:{Análisis físico financiero (AFF)}}
\textbf{Definición:} Planilla de control que trae el presupuesto completo, el avance mensual, gastos y la proyección de la obra. Porcentaje de avance que alimenta la facturación.
\subsection{\textcolor{teal}{Construcción}:{Análisis precio unitario (APU)}}
\textbf{Definición:} Corresponde al presupuesto detallado por partida (ítem único).
\subsection{\textcolor{teal}{Construcción}:{Anexo proveedores}}
\textbf{Definición:} Documento que se agrega al contrato original con los proveedores y corresponde generalmente a aumentos de contrato, diminuciones de contratos o contratación de nuevas actividades.
\subsection{\textcolor{teal}{Construcción}:{Arriendo de equipos y maquinarias I-Construye}}
\textbf{Definición:} Modulo habilitado para proveedores que permite arrendar equipos y maquinarias y sobre el periodo que lo tienen en arriendo generar la facturación mensual a cada proyecto.
\subsection{\textcolor{teal}{Construcción}:{Avance estados de pago obra}}
\textbf{Definición:} Corresponde al avance físico por partida (obra gruesa, terminaciones, gastos generales, etc.) respecto del presupuesto de la obra que genera estados de pago respectivos y mensuales.
\subsection{\textcolor{teal}{Construcción}:{Avance físico de obra}}
\textbf{Definición:} Progreso observado por el planificador de obra basado en el porcentaje de avance por actividad, por casa y por departamento-piso.
\subsection{\textcolor{teal}{Construcción}:{Benchmark obra gruesa}}
\textbf{Definición:} Indica cómo está la empresa en sus rendimientos de obra gruesa respecto del mercado. Si está en rojo, naranja y verde. El indicador se puede obtener por proyecto y por año.
\subsection{\textcolor{teal}{Construcción}:{Benchmark terminaciones}}
\textbf{Definición:} Indica cómo está la empresa en rendimiento de terminaciones respecto del mercado. Si está en rojo, naranja y verde. El indicador se puede obtener por proyecto y por año.
\subsection{\textcolor{teal}{Construcción}:{Carátula de pago}}
\textbf{Definición:} Corresponde a la presentación visual del estado de pago en un PDF que muestra de manera resumida el monto a pagar y corresponde a la información que se envía a contabilidad para generar la factura. En el caso de un proyecto, corresponde al avance edificado respecto al cual se hacen los cobros.
\subsection{\textcolor{teal}{Construcción}:{Centros de gestión IConstruye}}
\textbf{Definición:} Corresponde a un centro de costo basado en una unidad de negocio independiente a la otra y puede corresponder a un proyecto, un edificio, una gerencia, etc.
\subsection{\textcolor{teal}{Construcción}:{Código de producto iConstruye}}
\textbf{Definición:} Corresponde a la identificación del recurso, definido previamente en un maestro de recursos, conocido también como maestro de materiales (MM). El código de producto está asociado a una descripción, unidad de medida y una cuenta contable.
\subsection{\textcolor{teal}{Construcción}:{Convenios con proveedores}}
\textbf{Definición:} Corresponde al listado de productos en convenio con cada proveedor en cada zona y en cada empresa.
\subsection{\textcolor{teal}{Construcción}:{Costo contable proyecto}}
\textbf{Definición:} Corresponde a la suma de los gastos efectuados y facturados asociados al proyecto dentro del mes. Estos gastos se clasifican en gastos de materiales, gastos de contratos y subcontratos, gastos generales y gastos de la mano de obra.
\subsection{\textcolor{teal}{Construcción}:{Cotizaciones de materiales}}
\textbf{Definición:} Precio real entregado por el proveedor por un monto o cantidad de producto en un momento dado, con una vigencia limitada.
\subsection{\textcolor{teal}{Construcción}:{Cuadro de cotizaciones subcontratos}}
\textbf{Definición:} Corresponde a un cuadro comparativo de las cotizaciones a distintos proveedores respecto a un determinado insumo (pilas, anclajes, ascensores, basura, puertas, ventanas, muebles, cubiertas de muebles, impermeabilización, paisajismo, grupo electrógeno, equipos hidráulicos, corrientes débiles, electricidad, sanitario, clima, etc.)
\subsection{\textcolor{teal}{Construcción}:{Cubicación}}
\textbf{Definición:} Corresponde al procedimiento mediante el cual se calcula el volumen y la capacidad de un objeto o espacio. De ese modo, es posible estimar la cantidad de material necesario para desarrollar una edificación.
\subsection{\textcolor{teal}{Construcción}:{Cubicación de fierro}}
\textbf{Definición:} Corresponde al volumen en kilogramos de fierro por cierta superficie en metros cuadrados de un proyecto.
\subsection{\textcolor{teal}{Construcción}:{Cubicación de moldaje}}
\textbf{Definición:} Corresponde al volumen en metros cuadrados de moldaje por cierta superficie en metros cuadrados de un proyecto. Constructora Socovesa obtiene este dato de una consultora externa.
\subsection{\textcolor{teal}{Construcción}:{Cubicaciones de hormigón}}
\textbf{Definición:} Corresponde al volumen en metros cuadrados de hormigón por cierta superficie en metros cuadrados de un proyecto. Constructora Socovesa obtiene este dato de una consultora externa.
\subsection{\textcolor{teal}{Construcción}:{Cubicaciones de terminaciones}}
\textbf{Definición:} Corresponde al volumen de cierto material (que corresponda terminaciones) para cierta superficie en metros cuadrados de un proyecto.
\subsection{\textcolor{teal}{Construcción}:{Datos de identificación de proveedores}}
\textbf{Definición:} Datos de los proveedores publicados en el market place de I-Construye. Estos datos incluyen el RUT, razón social y representante legal de la empresa; y el nombre, apellido, teléfono y correo electrónico de los subcontratistas. Si el proveedor está integrado a la plataforma I-Construye, este puede administrar los datos de contacto que publica, mientras que sí no es el caso, cada empresa puede establecer sus contactos de proveedores, sin modificar RUT ni razón social.
\subsection{\textcolor{teal}{Construcción}:{Datos planos de respaldo de cubicaciones}}
\textbf{Definición:} Corresponden a los datos de superficies y cubicaciones de materiales por obra almacenados en archivos AutoCAD.
\subsection{\textcolor{teal}{Construcción}:{Desviación del ritmo de actividades}}
\textbf{Definición:} Indica como estoy desplazado de mi planificación inicial, como estoy al día de hoy respecto de mi día inicial y como debería terminar, siempre considerando unidades terminadas completas.
\subsection{\textcolor{teal}{Construcción}:{Elementos de imputación}}
\textbf{Definición:} Permite una estructura contable válida en SAP y tiene relación con aquellos elementos al que un centro de gestión puede asignar/imputar costos.
\subsection{\textcolor{teal}{Construcción}:{Estados de pago}}
\textbf{Definición:} El estado de pago corresponde al pago en forma parcial o total de los subcontratos adquiridos, dependiendo de avance y la conformidad de dicho contrato. Existen estados de pago anticipo, estados de pago avance y estados de pago retenidos.
\subsection{\textcolor{teal}{Construcción}:{Evaluaciones de subcontratistas}}
\textbf{Definición:} Corresponde a la nota final promedio por cada subcontratista de Almagro. Para ello se utilizan las evaluaciones de los profesionales de obra en base a diferentes criterios (calidad, solvencia técnica, seguridad, entre otros.). Sirve de utilidad para las licitaciones.
\subsection{\textcolor{teal}{Construcción}:{Factura}}
\textbf{Definición:} Una factura es un documento tributario emitido por el proveedor que refleja toda la información de una operación de compraventa y que da una obligación de pago en un plazo específico. La factura puede estar asociada a un estado de pago o a una guía de despacho.
\subsection{\textcolor{teal}{Construcción}:{Gastos del proyecto}}
\textbf{Definición:} Corresponde a la suma de los gastos consumidos asociados al proyecto dentro del mes. Estos gastos se clasifican en gastos de materiales, gastos de contratos y subcontratos, gastos generales y gastos de la mano de obra.
\subsection{\textcolor{teal}{Construcción}:{Gastos generales (GG)}}
\textbf{Definición:} Corresponde al gasto efectuado en la oficina central por conceptos de movilización, elementos de aseo, elementos de seguridad, ampolletas, focos gas, combustible, caja chica, cocina, etc. También se incluyen gastos como el personal técnico de obra; preparación de departamentos; personal auxiliar; consumos varios; maquinaria y equipos menores.
\subsection{\textcolor{teal}{Construcción}:{Gastos-ingresos departamentos internos constructora}}
\textbf{Definición:} Corresponde a los gastos e ingresos de los departamentos internos de la constructora, se incluyen los gastos de maquinarias, seguridad, paisajismo, distribución, bodega central entre otros.
\subsection{\textcolor{teal}{Construcción}:{Gestión de bodegas I-Construye}}
\textbf{Definición:} Módulo de I-Construye que apunta al control de inventario, control de recepciones y salidas de productos, para reflejar las existencias.
\subsection{\textcolor{teal}{Construcción}:{Guía de despacho}}
\textbf{Definición:} Documento con el que se hacen las recepciones de productos o servicios. 
\subsection{\textcolor{teal}{Construcción}:{Hombre día (HD)}}
\textbf{Definición:} Indica cuántos trabajadores realizan una determinada actividad por día de trabajo.
\subsection{\textcolor{teal}{Construcción}:{Indicador de terminalidad}}
\textbf{Definición:} Corresponde al número de unidades terminadas respecto al total de unidades que han sido iniciadas, Este indicador es realizado por el planificador y el encargado de calidad.
\subsection{\textcolor{teal}{Construcción}:{Indicador PSC}}
\textbf{Definición:} Con información de la mutual se genera este indicador psicosocial de los trabajadores.
\subsection{\textcolor{teal}{Construcción}:{Indicador seguridad y salud ocupacional (SSO)}}
\textbf{Definición:} Indicador entregado por la mutual respecto de la accidentabilidad y siniestralidad de los trabajadores de la empresa.
\subsection{\textcolor{teal}{Construcción}:{Indicador uso de tecnología obra}}
\textbf{Definición:} Indica cómo está la obra en tecnología respecto del mercado. Si está en rojo, naranja y verde. El indicador se puede obtener por proyecto y por año.
\subsection{\textcolor{teal}{Construcción}:{Indicadores de gestión documental I-Construye}}
\textbf{Definición:} Panel de indicadores con los colores rojo, amarillo, verde, dependiendo de cuántos documentos (órdenes de compra, facturas, etc.) tengo pendiente de aprobar.
\subsection{\textcolor{teal}{Construcción}:{Índice remoción de restricciones}}
\textbf{Definición:} Indica cuantas restricciones se levantaron en un periodo en una obra.
\subsection{\textcolor{teal}{Construcción}:{Informe variación de proyecciones de costos}}
\textbf{Definición:} Informe completo de las variaciones de costos consolidado para todos los proyectos. Indica alzas y bajas de la proyección, evolución de la proyección del proyecto.
\subsection{\textcolor{teal}{Construcción}:{Inventario}}
\textbf{Definición:} Listado de productos disponible en bodegas por cada uno de los centros de gestión de la empresa, se lista cantidad, precio promedio ponderado, código y descripción. Es el resultado de la suma de los ingresos menos las salidas que corresponden a las existencias.
\subsection{\textcolor{teal}{Construcción}:{Línea de balance de gastos generales (LOB GG)}}
\textbf{Definición:} Representación gráfica preliminar de las actividades de una obra, como una serie de líneas inclinadas que muestran la tasa de producción y que permiten detectar interferencias entre las distintas actividades. Incluye la ejecución de obra gruesa y terminaciones de forma genérica.
\subsection{\textcolor{teal}{Construcción}:{Línea de balance oficial (LOB)}}
\textbf{Definición:} Representación gráfica oficial de las actividades de una obra, como una serie de líneas inclinadas que muestran la tasa de producción y que permiten detectar interferencias entre las distintas actividades. Incluye la ejecución de obra gruesa y terminaciones de forma detallada.
\subsection{\textcolor{teal}{Construcción}:{Maestro de recursos - Arquitectura (IConstruye)}}
\textbf{Definición:} Listado de recursos asociado al área de arquitectura a nivel corporativo. Cada recurso tiene asociado un código, una descripción, una unidad de medida y una cuenta contable.
\subsection{\textcolor{teal}{Construcción}:{Maestro de recursos - Desarrollo (IConstruye)}}
\textbf{Definición:} Listado de recursos asociado al área de desarrollo a nivel empresa (Socovesa, Socovesa Sur y Almagro). Cada recurso tiene asociado un código, una descripción, una unidad de medida y una cuenta contable.
\subsection{\textcolor{teal}{Construcción}:{Maestro de recursos - Gastos de Administración y Ventas (IConstruye)}}
\textbf{Definición:} Listado de recursos asociado al área de administración y ventas a nivel corporativo. Cada recurso tiene asociado un código, una descripción, una unidad de medida y una cuenta contable.
\subsection{\textcolor{teal}{Construcción}:{Maestro de recursos - Maestro de materiales (MM)}}
\textbf{Definición:} Listado de códigos (recursos) asociado a algún área que permite estandarizar el proceso de compra de bienes y servicios. Cada maestro de recursos puede estar asociado a uno o más centros de gestión.
\subsection{\textcolor{teal}{Construcción}:{Maestro de recursos - Marketing (IConstruye)}}
\textbf{Definición:} Listado de recursos asociado al área de marketing a nivel corporativo. Cada recurso tiene asociado un código, una descripción, una unidad de medida y una cuenta contable.
\subsection{\textcolor{teal}{Construcción}:{Maestro de recursos - Sala de ventas (IConstruye)}}
\textbf{Definición:} Listado de recursos asociados a las salas de venta a nivel corporativo. Cada recurso tiene asociado un código, una descripción, una unidad de medida y una cuenta contable.
\subsection{\textcolor{teal}{Construcción}:{Mano de obra directa (M.O.D.)}}
\textbf{Definición:} Corresponde al valor presupuestado total de la mano de obra que es contratada por la constructora por proyecto.
\subsection{\textcolor{teal}{Construcción}:{Mano de obra subcontratada (M.O.S.)}}
\textbf{Definición:} Corresponde al valor presupuestado total de la mano de obra que se subcontrata por proyecto.
\subsection{\textcolor{teal}{Construcción}:{Matriz de terminalidad de actividades}}
\textbf{Definición:} La matriz de terminalidad indica el estado de cada departamento respecto a una actividad (foto del momento).
\subsection{\textcolor{teal}{Construcción}:{Nómina de pagos}}
\textbf{Definición:} Pago de los documentos tributarios emitidos por los proveedores que estan aprobados para poder ser pagados. La sumatoria de todas las facturas en los distintos centros de gestión se incorporan a través de esta nómina de pagos que se generan quincenal o mensualmente. Es el pago de la compañía a los proveedores.
\subsection{\textcolor{teal}{Construcción}:{Notas genéricas al presupuesto}}
\textbf{Definición:} Corresponden a las consideraciones del contrato presentado por la Constructora a la Inmobiliaria, respecto a obligaciones, permisos y otros detalles.
\subsection{\textcolor{teal}{Construcción}:{Notas técnicas al presupuesto}}
\textbf{Definición:} Corresponden a las consideraciones técnicas del presupuesto, tales como los valores pro forma (VPF) de subcontratos, conceptos que se consideran y no se consideran dentro de presupuesto; y otras consideraciones técnicas para obtener el valor del presupuesto.
\subsection{\textcolor{teal}{Construcción}:{Obra gruesa (OG)}}
\textbf{Definición:} Corresponde a la parte de una edificación que abarca desde los cimientos hasta la techumbre, incluida la totalidad de su estructura y muros divisorios, sin incluir las instalaciones, las terminaciones y cierres de vanos. Se incluye la enfierradura, los moldajes, el hormigón, entre otros.
\subsection{\textcolor{teal}{Construcción}:{Oficina de partes (electrónica)}}
\textbf{Definición:} Instancia electrónica para el repositorio de documentos electrónicos, incluyendo el módulo de facturacion electrónica.
\subsection{\textcolor{teal}{Construcción}:{Órdenes de compra}}
\textbf{Definición:} Documento que refleja un acuerdo comercial entre dos partes emitido por el comprador al proveedor.
\subsection{\textcolor{teal}{Construcción}:{Parámetro de costo mano de obra UF/M2 total}}
\textbf{Definición:} Costo total de proyecto en UF que corresponde a mano de obra.
\subsection{\textcolor{teal}{Construcción}:{Parámetro de costo UF/HD}}
\textbf{Definición:} Costo total del proyecto en UF por Hombre día (HD)
\subsection{\textcolor{teal}{Construcción}:{Parámetro de costo UF/M2 Edificio}}
\textbf{Definición:} Costo del proyecto total en UF por metro cuadrado edificio
\subsection{\textcolor{teal}{Construcción}:{Parámetro de costo UF/M2 total}}
\textbf{Definición:} Costo del proyecto total en UF por metro cuadrado total
\subsection{\textcolor{teal}{Construcción}:{Parámetro de costo UF/M2 Util}}
\textbf{Definición:} Costo del proyecto total en UF por metro cuadrado útil.
\subsection{\textcolor{teal}{Construcción}:{Parámetro de rendimiento HD/M2 Total}}
\textbf{Definición:} Costo total de proyecto en Hombre día (HD) que corresponde a mano de obra.
\subsection{\textcolor{teal}{Construcción}:{Parámetros mano de obra}}
\textbf{Definición:} Indica el costo de la mano de obra en UF o en hombre día (HD) por metro cuadrado. La información puede obtenerse por año, por proyecto, por etapa, por tipo de obra, entre otros.
\subsection{\textcolor{teal}{Construcción}:{Pedidos de materiales I-Construye}}
\textbf{Definición:} El pedido de materiales es usado para listar las necesidades de producto o materiales que son necesarios para realizar sus tareas y poder funcionar. Mientras no se genere una orden de compra, los pedidos de materiales se mantienen dentro de la plataforma I-Construye.
\subsection{\textcolor{teal}{Construcción}:{Perfiles IConstruye}}
\textbf{Definición:} Cada perfil tiene un listado determinado de roles que son los realizar distintas acciones dentro de la plataforma I-Construye.
\subsection{\textcolor{teal}{Construcción}:{Planificación de control de producción (PCP)}}
\textbf{Definición:} Corresponde al modo de planificación utilizado por las obras. Incluye la planificación semanal, planificación intermedia, desviación del ritmo, terminalidad y rendimientos.
\subsection{\textcolor{teal}{Construcción}:{Planificación intermedia de obra}}
\textbf{Definición:} Corresponde a la planificación de actividades de la obra para un periodo de doce semanas. La planificación es definida por el equipo de obra y el planificador.
\subsection{\textcolor{teal}{Construcción}:{Planificación semanal de obra}}
\textbf{Definición:} Corresponde a la planificación de actividades de la obra para un periodo semanal. La planificación es definida por el equipo de obra y el planificador.
\subsection{\textcolor{teal}{Construcción}:{Plazos avanzados/proyectados por actividad/obra}}
\textbf{Definición:} Control de avance de obras no rítmicas de proyecto (por ejemplo, fachadas y exteriores).
\subsection{\textcolor{teal}{Construcción}:{Plazos oficiales obra gruesa}}
\textbf{Definición:} Corresponde al periodo de tiempo estipulado de forma oficial para la ejecución de la obra gruesa de una obra. Es el establecido en la Línea de Balance Oficial (LOB).
\subsection{\textcolor{teal}{Construcción}:{Plazos oficiales terminaciones}}
\textbf{Definición:} Corresponde al periodo de tiempo estipulado de forma oficial para la ejecución de las terminaciones de una obra. Es el establecido en la Línea de Balance Oficial (LOB).
\subsection{\textcolor{teal}{Construcción}:{Plazos preliminares obra gruesa}}
\textbf{Definición:} Corresponde al periodo de tiempo estipulado de forma anticipada para la ejecución de la obra gruesa de una obra. Es el establecido en la Línea de Balance de Gastos Generales (LOB GG)
\subsection{\textcolor{teal}{Construcción}:{Plazos preliminares terminaciones}}
\textbf{Definición:} Corresponde al periodo de tiempo estipulado de forma anticipada para la ejecución de las terminaciones de una obra. Es el establecido en la Línea de Balance de Gastos Generales (LOB GG)
\subsection{\textcolor{teal}{Construcción}:{Presupuesto}}
\textbf{Definición:} Cálculo anticipado del costo de una obra o un servicio.
\subsection{\textcolor{teal}{Construcción}:{Presupuesto entibaciones}}
\textbf{Definición:} Cálculo anticipado del costo del socalzado de la obra, que corresponde a movimientos de tierra y sistemas de entibaciones ("soil nailing", pilas o pilotes).
\subsection{\textcolor{teal}{Construcción}:{Presupuesto inmobiliaria}}
\textbf{Definición:} Corresponde al costo presupuestado en UF para el proyecto más la utilidad de la inmobiliaria
\subsection{\textcolor{teal}{Construcción}:{Presupuesto obra}}
\textbf{Definición:} Corresponde al costo presupuestado en UF para el proyecto
\subsection{\textcolor{teal}{Construcción}:{Presupuesto obra gruesa}}
\textbf{Definición:} Cálculo anticipado del costo de la obra gruesa de un proyecto (enfierradora, moldajes, hormigón, etc.)
\subsection{\textcolor{teal}{Construcción}:{Presupuesto obras extras}}
\textbf{Definición:} Presupuesto por conceptos adicionales como cambios en las especificaciones de la obra, acuerdos de ventas o solicitudes de clientes.
\subsection{\textcolor{teal}{Construcción}:{Presupuesto obras preliminares}}
\textbf{Definición:} Cálculo anticipado del costo de las obras que se dan al inicio de un proyecto, por ejemplo: cierres provisorios, instalación de faena, demoliciones, etc.
\subsection{\textcolor{teal}{Construcción}:{Presupuesto terminaciones}}
\textbf{Definición:} Cálculo anticipado de costo de las terminaciones de las obras.
\subsection{\textcolor{teal}{Construcción}:{Proveedores integrados a I-Construye}}
\textbf{Definición:} Corresponde a aquellos proveedores enrolados en la plataforma de I-Construye y que forman parte de la comunidad de empresas en el market-place de I-Construye.
\subsection{\textcolor{teal}{Construcción}:{Proyección de costos por familia}}
\textbf{Definición:} Corresponde al cierre en UF del costo por concepto de gastos generales, maquinarias y equipos, materiales, subcontratos y mano de obra (familias de costos). La diferencia entre el costo presupuesto y la proyección del costo generan ahorros o mayores costos (pérdidas).
\subsection{\textcolor{teal}{Construcción}:{Proyección de termino de actividad/obra}}
\textbf{Definición:} Proyección que se hace en base a los ritmos reales de la obra indicando cuando va a terminar una actividad y como interfiere con sus actividades predecesoras y sucesoras.
\subsection{\textcolor{teal}{Construcción}:{Recepción de materiales}}
\textbf{Definición:} Es un documento asociado a la orden de compra que indica esta última se hace efectiva.
\subsection{\textcolor{teal}{Construcción}:{Rendimiento de actividades críticas (m2/hd)}}
\textbf{Definición:} Indica cuántos metros cuadrados avanza un trabajador por día (hombre-día) en actividades críticas (tabiques, cerámica, empastes y pinturas).
\subsection{\textcolor{teal}{Construcción}:{Rendimiento mano de obra por especialidad (obra gruesa y terminaciones)}}
\textbf{Definición:} Cantidad de unidades ejecutadas en un día por cada especialidad (jornal, maestro, pintor, etc.). La información puede obtenerse por año, por proyecto, por tipo de obra (obra gruesa, terminaciones) entre otros.
\subsection{\textcolor{teal}{Construcción}:{Rendimiento obra gruesa (m2/hd)}}
\textbf{Definición:} Indicador metro cuadrado por hombre día en actividades de obra gruesa (por ejemplo, moldaje)
\subsection{\textcolor{teal}{Construcción}:{Reportes I-Construye}}
\textbf{Definición:} En I-Construye se generan reportes para las órdenes de compra y pedidos de materiales, contratos y estados de pago (reportes subcontratos), movimientos de bodega (reportes bodega), control pago y documentos tributarios recibidos (reportes de facturación) y reportes de facturación electrónica.
\subsection{\textcolor{teal}{Construcción}:{Resultados mano de obra}}
\textbf{Definición:} Resultados generales relacionados con la mano de obra en UF o en hombre día presupuestada, proyectada y sus diferencias. La información puede obtenerse por año, por proyecto, por etapa, por tipo de obra, entre otros.
\subsection{\textcolor{teal}{Construcción}:{Retenciones provisionadas por obra}}
\textbf{Definición:} Cuanto es lo que tiene cada obra en retención de devoluciones.
\subsection{\textcolor{teal}{Construcción}:{Ritmo de avance de obra}}
\textbf{Definición:} Casa/departamentos semanales. En caso de obra gruesa de edificios, pisos mensuales.
\subsection{\textcolor{teal}{Construcción}:{Roles IConstruye}}
\textbf{Definición:} Listado de definiciones asociadas a las funciones del usuario, se agrupan en perfiles que dependen del cargo del usuario. Por ejemplo, tengo el perfil de administrador de obra y de jefe de bodega con distintos rolaes cada uno dependiendo de las funciones de su cargo.
\subsection{\textcolor{teal}{Construcción}:{Subcontrato proveedores}}
\textbf{Definición:} Contratos entre el mandante y el subcontratista para realizar una determinada labor, con ciertas clausulas como avances, retenciones o anticipos de ambas partes.
\subsection{\textcolor{teal}{Construcción}:{Terminaciones y revestimientos}}
\textbf{Definición:} Corresponden a todas aquellas obras definitivas que, a diferencia de la obra gruesa, no poseen un carácter estructural, pero son necesarias para el buen funcionamiento de ella. Son arreglos para darle finalización a los detalles, aportando un aspecto estético y habitable a toda la obra.
\subsection{\textcolor{teal}{Construcción}:{Usuarios IConstruye}}
\textbf{Definición:} Usuarios creados dentro de la plataforma I-Construye que pueden ser operativos o de consulta y que se asocian a uno o más centros de gestión. Asociado a un usuario se encuentran los perfiles.
\subsection{\textcolor{teal}{Construcción}:{Valor ganado obra}}
\textbf{Definición:} Corresponde al valor esperado de lo que deberías tener gastado de acuerdo al estado de avance de estado de pago.
\subsection{\textcolor{teal}{Construcción}:{Valor proforma (VPF)}}
\textbf{Definición:} Estimación del precio de un objeto, conjunto de objetos, ítem o partida de un presupuesto cuya determinación exacta y precisa no es posible de establecer en la etapa de estudio del precio de un proyecto.
\section{\textcolor{blue}{Contabilidad}}
\subsection{\textcolor{teal}{Contabilidad}:{Flujo de aprobación - Rindegastos}}
\textbf{Definición:} Está definido a partir de una política y tiene relación con la aprobación (usuarios aprobadores) del gasto a quienes rinden (usuarios rendidores).
\subsection{\textcolor{teal}{Contabilidad}:{Fondo fijo - Rindegasto}}
\textbf{Definición:} Corresponde a la cantidad de dinero asignada a un usuario por concepto de caja chica, para solventar gastos necesarios del área. El fondo fijo a cada usuario está reflejado en SAP.
\subsection{\textcolor{teal}{Contabilidad}:{Informe de gastos - Rindegastos}}
\textbf{Definición:} Es la sumatoria de cada gasto rendido por usuario que detalla para cada gasto el monto, el detalle, proveedor, empresa, etc.
\subsection{\textcolor{teal}{Contabilidad}:{Política de gastos - Rindegastos}}
\textbf{Definición:} Cada política de gastos tiene configuraciones propias asociadas a código, cuenta, elementos de imputación, tipos de documentos, usuarios, etc. y corresponde al punto de acceso para que uno o varios usuarios puedan rendir los gastos en una determinada área.
\subsection{\textcolor{teal}{Contabilidad}:{Rendidores - Rindegastos}}
\textbf{Definición:} Usuario que tiene un fondo fijo asignado y que por tal monto debe rendir lo gastado. (puede rendir hasta el monto asignado)
\subsection{\textcolor{teal}{Contabilidad}:{UF}}
\textbf{Definición:} Unidad de Fomento que reajusta con el IPC acumulado del mes. Fuentes Banco Central
\section{\textcolor{blue}{Finanzas}}
\subsection{\textcolor{teal}{Finanzas}:{AFR}}
\textbf{Definición:} significa aporte financiero reembolsable, nosotros manejamos Pagarés de agua (Aguas Andinas, Esval, entre otras). Representan un crédito que entrega el urbanizador a la empresa sanitaria, con el fin de financiar los costos adicionales que genera una mayor demanda, producto del proyecto inmobiliario y de la mayor infraestructura que se requiere para atenderla. Los aportes se otorgan por un plazo máximo de 15 años.
\subsection{\textcolor{teal}{Finanzas}:{Amortización a Deuda}}
\textbf{Definición:} es cuando damos la instrucción al ejecutivo de cuenta corriente de un banco en particular a cargar nuestra cuenta para abonar dineros a un crédito que indiquemos para así disminuir el saldo capital del mismo. 
\subsection{\textcolor{teal}{Finanzas}:{Aprobación Bancaria}}
\textbf{Definición:} Respuesta del banco a solicitud de cliente por instrumento financiero (crédito) que el banco desea destinar a sus clientes que cuentan con buen historial y capacidad de pago. El cliente debe presentar documentos que acrediten sus condiciones.
\subsection{\textcolor{teal}{Finanzas}:{Banco Alzante}}
\textbf{Definición:} Entidad que financia construcción de un proyecto para la venta o que tiene una garantía hipotecaria sobre las propiedades. 
\subsection{\textcolor{teal}{Finanzas}:{BCU/BCP}}
\textbf{Definición:} El Banco Central de Chile emite instrumentos de deuda, tales como bonos y pagarés con diferentes denominaciones, con plazos que van desde los 30 días a 20 años, y que pueden ser emitidos en pesos o en unidades de fomento: BCU (Banco Central Bonos Bullet en UF) o BCP (Banco Central Bonos Bullet en Pesos).
\subsection{\textcolor{teal}{Finanzas}:{Boleta de Garantía}}
\textbf{Definición:} operación financiera en la cual el banco emite un documento que resguarda obligaciones que la empresa emisora contrae con un tercero (beneficiario de la boleta).
\subsection{\textcolor{teal}{Finanzas}:{Boleta de garantía pagadera a 30 días}}
\textbf{Definición:} Boleta de garantía que es pagada transcurridos 30 días desde el cobro de esta en el banco donde fue emitida (es la más utilizada en Socovesa).
\subsection{\textcolor{teal}{Finanzas}:{Boleta de garantía pagadera a la vista}}
\textbf{Definición:} Boleta de garantía que es pagada al momento de ser presentada físicamente en el banco donde fue emitida.
\subsection{\textcolor{teal}{Finanzas}:{Borrador}}
\textbf{Definición:} ver Compraventa
\subsection{\textcolor{teal}{Finanzas}:{Cambio Oferente}}
\textbf{Definición:} Cambio de titular de promesa de compraventa.
\subsection{\textcolor{teal}{Finanzas}:{Carta de crédito}}
\textbf{Definición:} Es un compromiso de un banco a nombre de un comprador (cliente importador, en nuestro caso Empresas Socovesa) de pagar al vendedor (beneficiario exportador) una cantidad acordada a cambio de que el vendedor entregue los documentos y mercancías requeridas en una fecha determinada (por ejemplo ascensores).
\subsection{\textcolor{teal}{Finanzas}:{Carta de resguardo}}
\textbf{Definición:} Documento escrito, que consiste en la obligación que asume un banco que otorga el crédito a un cliente, de cancelar la totalidad de la deuda que esta tiene en otra institución acreedora.
\subsection{\textcolor{teal}{Finanzas}:{Carta Pie}}
\textbf{Definición:} Documento con el detalle de lo pagado por el cliente a la Inmobiliaria que se le entrega al banco que financiará la compra y a solicitud de la institución.
\subsection{\textcolor{teal}{Finanzas}:{Carta Resguardo}}
\textbf{Definición:} documento que consiste en la obligación que asume un banco que otorga el crédito a un cliente, de cancelar la totalidad de la deuda que esta tiene en otra institución acreedora.
\subsection{\textcolor{teal}{Finanzas}:{Cesión de derechos}}
\textbf{Definición:} decisión que toma un propietario sobre sus bienes con el objetivo de transferir un derecho a otra persona quien será el nuevo poseedor. Normalmente se refiere a la acción que se hace cuando hay cambio de oferente en la promesa de compraventa.
\subsection{\textcolor{teal}{Finanzas}:{Cierre de Condiciones}}
\textbf{Definición:} Cuando se definen las condiciones específicas (monto, tasa, plazo, porcentaje de financiamiento, etc.) del crédito que otorga el banco para aceptación del cliente y previo a la confección de borrador.
\subsection{\textcolor{teal}{Finanzas}:{Codeudor}}
\textbf{Definición:} Es quien comparte una deuda con otra persona en igualdad de condiciones, por lo que el acreedor podrá cobrar directamente al deudor principal o a quien solidarizó
\subsection{\textcolor{teal}{Finanzas}:{Compañía de Seguros}}
\textbf{Definición:} Es una empresa que se encarga de asegurar riesgos a terceros, de tal manera que protege o resguarda los bienes materiales de los riesgos a los que estos están expuestos.
\subsection{\textcolor{teal}{Finanzas}:{Compensación}}
\textbf{Definición:} Es un excedente recibido, dado que en los instrumentos financieros tomados (forward o swaps) sobre el vencimiento de un crédito tuvieron saldo a favor para nosotros.
\subsection{\textcolor{teal}{Finanzas}:{Conservador de Bienes y Raíces o CBR}}
\textbf{Definición:} Los Conservadores de Bienes Raíces son ministros de fe encargados de los registros conservatorios de bienes raíces cuyo objeto principal es mantener la historia de la propiedad inmueble y otorgar una completa publicidad a los gravámenes que pueden afectar a los bienes raíces.
\subsection{\textcolor{teal}{Finanzas}:{Copia Maestra}}
\textbf{Definición:} Copia de la compraventa con al menos la firma de una de las partes.
\subsection{\textcolor{teal}{Finanzas}:{Copropiedad}}
\textbf{Definición:} Reglamento, Planos, Certificado
\subsection{\textcolor{teal}{Finanzas}:{Corredor de Seguros}}
\textbf{Definición:} Experto en seguros, con cuya experiencia permite que se desempeñe como el Intermediador entre el cliente y las compañías de seguros. Muchas veces prestan servicios personalizados a sus contratantes.
\subsection{\textcolor{teal}{Finanzas}:{Costear/ Cerrar un proyecto}}
\textbf{Definición:} Asignar costo a las unidades de los proyectos terminados.
\subsection{\textcolor{teal}{Finanzas}:{Cotitular}}
\textbf{Definición:} es responsable de la deuda de igual forma que lo es el titular del préstamo, y debe responder ante el préstamo con su dinero o sus propiedades.
\subsection{\textcolor{teal}{Finanzas}:{Crédito Hipotecario}}
\textbf{Definición:} herramienta que Institución Financiera proporciona, con la finalidad de entregar financiamiento para la compra de un inmueble, sea este nuevo o usado.
\subsection{\textcolor{teal}{Finanzas}:{Curse}}
\textbf{Definición:} cuando la operación se encuentra aprobada y cumple los requisitos para la elaboración del borrador de compraventa por el banco.
\subsection{\textcolor{teal}{Finanzas}:{Declaración Personal de Salud o DPS}}
\textbf{Definición:} Manifestación del asegurado en la propuesta o solicitud de seguro, de su condición y estado de salud y enfermedades preexistentes, para que la Compañía de Seguros decida sobre la aceptación y tarificación del riesgo.
\subsection{\textcolor{teal}{Finanzas}:{Depósito a Plazo}}
\textbf{Definición:} es un producto de inversión de renta fija y bajo riesgo, ya que, desde un principio, conocerás cuál será tu ganancia de acuerdo al monto que inviertas y el plazo que elijas, pueden ser en pesos, UF o dólares. Se utiliza normalmente en las compras de terrenos.
\subsection{\textcolor{teal}{Finanzas}:{Desistimiento}}
\textbf{Definición:} dejar sin efecto el contrato celebrado, notificándose a la otra parte. Normalmente se refiere al no cumplimiento de la promesa de compraventa. 
\subsection{\textcolor{teal}{Finanzas}:{Desistimiento de cobro}}
\textbf{Definición:} En el caso de las garantías, corresponde a desistir del cobro de estas, las cuales ya habían sido presentadas a cobro en la institución financiera.
\subsection{\textcolor{teal}{Finanzas}:{Escritura de Alzamiento}}
\textbf{Definición:} Documento que comprueba que el inmueble o terreno se encuentra libre de cualquier deuda u obligación.
\subsection{\textcolor{teal}{Finanzas}:{Estado Resultados  EERR}}
\textbf{Definición:} resultado de negocios en un plazo determinado
\subsection{\textcolor{teal}{Finanzas}:{Estudio de títulos}}
\textbf{Definición:} Evaluación que realiza un abogado de todos los antecedentes legales de una propiedad o terreno que es objeto de venta o que se recibiría por parte de un banco como hipoteca.
\subsection{\textcolor{teal}{Finanzas}:{Flujo de aprobación (Rindegastos)}}
\textbf{Definición:} Está definido a partir de una política y tiene relación con la aprobación (usuarios aprobadores) del gasto a quienes rinden (usuarios rendidores).
\subsection{\textcolor{teal}{Finanzas}:{Flujo Largo de Caja}}
\textbf{Definición:} Corresponde al Saldo de Caja proyectado a dos o tres años.
\subsection{\textcolor{teal}{Finanzas}:{Fondos fijos (Rindegastos)}}
\textbf{Definición:} Corresponde a la cantidad de dinero asignada a un usuario por concepto de caja chica, para solventar gastos necesarios del área. El fondo fijo a cada usuario está reflejado en SAP.
\subsection{\textcolor{teal}{Finanzas}:{Fondos Mutuos}}
\textbf{Definición:} Instrumento financiero en el cual invertimos los excedentes diarios, es de bajo riesgo y es utilizado comúnmente en Socovesa para mantener liquidez rápida entre los bancos de una misma empresa mediante los rescates.
\subsection{\textcolor{teal}{Finanzas}:{Forward}}
\textbf{Definición:} instrumento financiero utilizado para fijar el tipo de cambio futuro.
\subsection{\textcolor{teal}{Finanzas}:{Glosa boleta garantía}}
\textbf{Definición:} Es donde se debe señalar la caución (motivo) para lo cual se está solicitando la boleta de garantía.
\subsection{\textcolor{teal}{Finanzas}:{GP}}
\textbf{Definición:} Certificado que indica si una propiedad tiene gravámenes como hipotecas, usufructos, servidumbres y prohibiciones para vender.
\subsection{\textcolor{teal}{Finanzas}:{Hipoteca}}
\textbf{Definición:} Garantía que se deja a una institución financiera para asegurar el cumplimiento de una obligación.
\subsection{\textcolor{teal}{Finanzas}:{IMACEC}}
\textbf{Definición:} El Índice Mensual de Actividad Económica, es una estimación que resume la actividad de los distintos sectores de la economía en un determinado mes, a precios del año anterior; su variación interanual constituye una aproximación de la evolución del PIB.
\subsection{\textcolor{teal}{Finanzas}:{Índices Inmobiliarios}}
\textbf{Definición:} Principales Índices relacionados a la industria (Venta, Recuperaciones, Stock, Meses para Agotar Stock, etc) los cuales tienen carácter estratégico.
\subsection{\textcolor{teal}{Finanzas}:{Inflación}}
\textbf{Definición:} Aumento generalizado y sostenido de los precios de los bienes y servicios existentes en el mercado durante un período de tiempo.
\subsection{\textcolor{teal}{Finanzas}:{Informe de gastos (Rindegastos)}}
\textbf{Definición:} Es la sumatoria de cada gasto rendido por usuario que detalla para cada gasto el monto, el detalle, proveedor, empresa, etc. (documento que respalda gastos)
\subsection{\textcolor{teal}{Finanzas}:{Instrucciones Notariales}}
\textbf{Definición:} documento legal en la cual se indica al Notario que guarde custodia de algún documento. Estos documentos son generalmente cheques, vale vista o depósito a plazo, y el Notario los resguarda hasta que se cumplan ciertas condiciones. 
\subsection{\textcolor{teal}{Finanzas}:{Lanzamiento}}
\textbf{Definición:} Hito en el que un proyecto comienza la venta.
\subsection{\textcolor{teal}{Finanzas}:{Leasing habitacional}}
\textbf{Definición:} se conoce como “arriendo con compromiso de compra”, donde se establece un contrato que permite hacer uso exclusivo de la vivienda hasta pagar la última cuota, que funciona como un dividendo. Una vez cumplido el plazo, el deudor pasa a ser propietario.
\subsection{\textcolor{teal}{Finanzas}:{Leasing operativo}}
\textbf{Definición:} Es un arrendamiento de vehículos por el plazo de 12 meses o más. En la cuota mensual se incluyen los servicios de mantenimiento, averías, cambio de neumáticos, pagos de impuestos y de seguros, etc. Sirve para mantener una flota sin tener las preocupaciones de trámites legales, administración de documentos, entre otros.
\subsection{\textcolor{teal}{Finanzas}:{Liquidación de crédito}}
\textbf{Definición:} Documento financiero en el cual se especifican los datos de un crédito hipotecario (la forma y día de pago, datos del deudor, datos del beneficiario, monto de la operación, etc).
\subsection{\textcolor{teal}{Finanzas}:{Liquidador de Seguros}}
\textbf{Definición:} Empresas nombradas por las compañías de seguros para determinar la procedencia de la cobertura y monto de los daños en función de las condiciones de las pólizas.
\subsection{\textcolor{teal}{Finanzas}:{Márgenes Promesados}}
\textbf{Definición:} Porcentaje estimado de ganancia que se genera por la venta de una unidad, específicamente al momento de firmar la promesa de compraventa.
\subsection{\textcolor{teal}{Finanzas}:{Novación}}
\textbf{Definición:} Es cuando uno o varios créditos adquiridos en un banco son portadas a otro constituyendo un nuevo crédito y evitando el pago de impuesto de timbres nuevamente.
\subsection{\textcolor{teal}{Finanzas}:{Pactos Financieros}}
\textbf{Definición:} Es una operación de compraventa al contado de instrumentos financieros, realizada en forma conjunta y simultánea con una compraventa a plazo sobre los mismos o sobre otros instrumentos equivalentes, que las partes hayan acordado como sustitutos de los primeros, es utilizado comúnmente en Socovesa para mantener liquidez rápida entre los bancos de una misma empresa mediante los rescates.
\subsection{\textcolor{teal}{Finanzas}:{Pagaré}}
\textbf{Definición:} documento mediante el cual adquirimos una obligación con un banco. Se utiliza como respaldo de un crédito.
\subsection{\textcolor{teal}{Finanzas}:{PIB}}
\textbf{Definición:} El Producto Interno Bruto es una medida del valor de la actividad económica de un país. Básicamente calcula cuál fue la producción en bienes y servicios que se hizo en un periodo de tiempo específico.
\subsection{\textcolor{teal}{Finanzas}:{Política de gastos (Rindegastos)}}
\textbf{Definición:} Cada política de gastos tiene configuraciones propias asociadas a código, cuenta, elementos de imputación, tipos de documentos, usuarios, etc.; y corresponde al punto de acceso para que uno o varios usuarios puedan rendir los gastos en una determinada área.
\subsection{\textcolor{teal}{Finanzas}:{Póliza de Seguro}}
\textbf{Definición:} Es un contrato entre un contratante y una compañía de seguros, que establece los derechos y obligaciones de ambos, en relación al seguro contratado.
\subsection{\textcolor{teal}{Finanzas}:{Póliza de Urbanización}}
\textbf{Definición:} Es un documento financiero que emiten las compañías de seguros, para cubrir al beneficiario de las pérdidas en dinero que cause el incumplimiento de la obligación por parte del contratista (quien toma la póliza).
\subsection{\textcolor{teal}{Finanzas}:{Post Venta}}
\textbf{Definición:} El Servicio Post-Venta es la acción de seguir prestando servicio al cliente, es decir, tras realizar la venta al cliente, hay que seguir manteniendo una relación con dicho cliente, y, es de máxima importancia al igual que el proceso anterior a la venta.
\subsection{\textcolor{teal}{Finanzas}:{Preaprobación bancaria}}
\textbf{Definición:} Evaluación que da el banco a clientes que cuentan con buen historial y capacidad de pago, sin revisión de documentos, donde se define que dados los datos entregados por el cliente se le otorgaría un crédito hipotecario según las condiciones definidas (monto, % de financiamiento y plazo).
\subsection{\textcolor{teal}{Finanzas}:{Real Proyectado (RP)}}
\textbf{Definición:} Informe que muestra costos y gastos reales de las obras y proyecta los mismos al término de su construcción.
\subsection{\textcolor{teal}{Finanzas}:{Recepción Municipal (RM)}}
\textbf{Definición:} Es el trámite mediante el cual se recibe una obra de construcción con el fin de autorizar su uso.
\subsection{\textcolor{teal}{Finanzas}:{Redenominación}}
\textbf{Definición:} Es cuando un crédito cambia de moneda pasando por ejemplo de UF a pesos.
\subsection{\textcolor{teal}{Finanzas}:{Rendidores (Rindegastos)}}
\textbf{Definición:} Usuario que tiene un fondo fijo asignado y que por tal monto debe rendir lo gastado. (puede rendir hasta el monto asignado)
\subsection{\textcolor{teal}{Finanzas}:{Renovación}}
\textbf{Definición:} es cuando pagamos intereses por un crédito adquirido y pactamos un nuevo vencimiento. Este pago de intereses no reduce el saldo capital del crédito. En esta figura se mantiene la moneda, el banco y el saldo capital, lo único que cambia es el interés que ofrece el banco a un plazo determinado.
\subsection{\textcolor{teal}{Finanzas}:{Reparos}}
\textbf{Definición:} Antecedentes pendientes en un estudio de títulos.
\subsection{\textcolor{teal}{Finanzas}:{Saldo Caja}}
\textbf{Definición:} Es el saldo con el cual contamos como disponible, previo descuento de las obligaciones que debemos cubrir durante el día (pago proveedores, subcontratos, cheques, remuneraciones, impuestos, intereses, entre otras) y descontando previamente las inversiones
\subsection{\textcolor{teal}{Finanzas}:{Seguro de AP}}
\textbf{Definición:} El seguro de Accidentes Personales es una protección para el asegurado y su entorno, en caso de accidente. La persona que contrata este tipo de seguro es indemnizado en caso de incapacidad total y permanente o, incluso, fallecimiento.
\subsection{\textcolor{teal}{Finanzas}:{Seguro de Equipos Móviles}}
\textbf{Definición:} Póliza que cubre todos los riesgos de un equipo móvil (tractor, bobcat, grúa, grupo electrógeno, etc).
\subsection{\textcolor{teal}{Finanzas}:{Seguro de garantía de Venta en Verde}}
\textbf{Definición:} Permite a los futuros compradores de inmuebles en construcción, tener la seguridad de que en el caso que la inmobiliaria o constructora no cumpla con sus obligaciones, la Compañía le devolverá los dineros que hubiesen entregado a modo de anticipo.
\subsection{\textcolor{teal}{Finanzas}:{Seguro de Incendio}}
\textbf{Definición:} Esta póliza cubre todos los riesgos de un proyecto (o bien raíz) terminado (después de la RM).
\subsection{\textcolor{teal}{Finanzas}:{Seguro de RC}}
\textbf{Definición:} El seguro de Responsabilidad Civil protege al asegurado en caso de ser declarado responsable por haber causado daños a un tercero, ya sea a la persona o a sus bienes. Este seguro entraría a operar en el evento en que el asegurado deba reparar el daño causado.
\subsection{\textcolor{teal}{Finanzas}:{Seguro desgravamen}}
\textbf{Definición:} es aquel que permite, en caso de muerte del deudor, extinguir o eliminar la deuda pendiente con el acreedor (banco). Su contratación por ley es obligatoria para las personas que adquieren un crédito hipotecario.
\subsection{\textcolor{teal}{Finanzas}:{Seguro TRC}}
\textbf{Definición:} Seguro de Todo Riesgo Construcción. Esta póliza cubre todos los riesgos presentes durante la construcción de una obra (hasta la RM).
\subsection{\textcolor{teal}{Finanzas}:{Seguros de Vehículos}}
\textbf{Definición:} Póliza que cubre todos los riesgos de un vehículo (camioneta, camión, furgón).
\subsection{\textcolor{teal}{Finanzas}:{Set de finanzas}}
\textbf{Definición:} Conjunto de informes que se envían a las filiales para su presentación en el directorio.
\subsection{\textcolor{teal}{Finanzas}:{Swaps}}
\textbf{Definición:} Instrumento financiero utilizado para el intercambio de flujos de pago en el futuro.
\subsection{\textcolor{teal}{Finanzas}:{TAB}}
\textbf{Definición:} Tasas activas bancarias (TAB), que buscan reflejar el costo de fondos de las instituciones financieras.
\subsection{\textcolor{teal}{Finanzas}:{Tasa Nominal}}
\textbf{Definición:} Tasa de interés de un crédito en pesos
\subsection{\textcolor{teal}{Finanzas}:{Tasación}}
\textbf{Definición:} Es el informe de valorización de un inmueble realizado bajo requisitos marcados por ley. El fin de la tasación hipotecaria es servir de garantía para el préstamo Hipotecario.
\subsection{\textcolor{teal}{Finanzas}:{Titular}}
\textbf{Definición:} sujeto que puede ejercer, promover y exigir un derecho hipotecario
\subsection{\textcolor{teal}{Finanzas}:{TPM}}
\textbf{Definición:} Es la tasa que ocupa el Banco Central de Chile como instrumento para llevar a cabo la Política Monetaria. En la actualidad, el objetivo de la Política Monetaria es mantener la inflación estable con un valor meta de 3% y rango de variación entre 2% y 4%, en un horizonte de 24 meses.
\subsection{\textcolor{teal}{Finanzas}:{Vale vista}}
\textbf{Definición:} se trata de un documento del banco que representa la existencia de dinero en efectivo; este se emite bajo órdenes de las instituciones financieras por cuenta de los clientes. Se utiliza normalmente en las compras de terrenos.
\subsection{\textcolor{teal}{Finanzas}:{Vencimiento de Intereses}}
\textbf{Definición:} fecha en la cual debemos cancelar los intereses de un crédito que están determinados por un plazo, una tasa y un saldo de capital.
\subsection{\textcolor{teal}{Finanzas}:{Visado}}
\textbf{Definición:} etapa de revisión de documentos en el banco previo etapa de curse.
\section{\textcolor{blue}{Gobierno de Datos}}
\subsection{\textcolor{teal}{Gobierno de Datos}:{Administración de Bases de Datos (Database Management)}}
\textbf{Definición:} La gestión de bases de datos se refiere a las acciones que realiza una empresa, un área o una persona para manipular y controlar los datos a fin de cumplir las condiciones necesarias durante todo el ciclo de vida de los datos.
\subsection{\textcolor{teal}{Gobierno de Datos}:{Almacén de Datos (Data Warehouse)}}
\textbf{Definición:} Sistema que agrega y combina información de diferentes fuentes en un repositorio de datos único y centralizado; consistente para respaldar el análisis empresarial, la minería de datos, inteligencia artificial y aprendizaje automático. No solo incluye datos, sino herramientas, procedimientos, capacitaciones y otras facilidades que facilitan el acceso a los datos para aquellos que toman decisiones.
\subsection{\textcolor{teal}{Gobierno de Datos}:{Almacenamiento de Datos (Data Storage)}}
\textbf{Definición:} Conjunto de especificaciones que sirven para definir cómo, cuándo y qué información se almacena.
\subsection{\textcolor{teal}{Gobierno de Datos}:{Aprendizaje Automático (Machine Learning)}}
\textbf{Definición:} Subcampo de la inteligencia artificial que permite que las máquinas aprendan de datos o experiencias pasadas sin ser programadas explícitamente.
\subsection{\textcolor{teal}{Gobierno de Datos}:{Aprendizaje Profundo (Deep Learning)}}
\textbf{Definición:} Subconjunto del aprendizaje automático donde las redes neuronales artificiales, algoritmos inspirados en el cerebro humano, aprenden de grandes cantidades de datos.
\subsection{\textcolor{teal}{Gobierno de Datos}:{Arquitectura de Datos (Data Architecture)}}
\textbf{Definición:} Corresponde a un marco de cómo la infraestructura de TI respalda su estrategia de datos. Tiene como objetivo visualizar cómo se adquieren, transportan, almacenan, consultan y protegen los datos.
\subsection{\textcolor{teal}{Gobierno de Datos}:{Base de Datos (Database)}}
\textbf{Definición:} Conjunto de datos almacenados y estructurados según sus características o tipología para ser utilizados o consultados con posterioridad.
\subsection{\textcolor{teal}{Gobierno de Datos}:{Calidad del Dato (Data quality)}}
\textbf{Definición:} Mide hasta qué nivel los datos cumplen el propósito esperado, considerando que el objetivo principal de recopilar datos es tener información útil y confiable para la toma de decisiones.
\subsection{\textcolor{teal}{Gobierno de Datos}:{Catalogación de Datos (Data Cataloging)}}
\textbf{Definición:} Es el proceso de hacer un inventario organizado de sus datos. Una vez que haya completado su proceso de mapeo de datos, el catálogo de datos es lo que usará para indexar dónde se almacena todo.
\subsection{\textcolor{teal}{Gobierno de Datos}:{Ciencia de Datos (Data Science)}}
\textbf{Definición:} Esta compuesto por métodos científicos en los que se utilizan algoritmos, estadísticas, procesos, sistemas, ingeniería software para obtener conocimiento, resolver problemas analíticos y tener un mejor entendimiento de la información. Los científicos de datos extraen la información que se utiliza posteriormente en los negocios para mejorar las estrategias.
\subsection{\textcolor{teal}{Gobierno de Datos}:{COBIT}}
\textbf{Definición:} COBIT (Control Objetives for Information and Related Technology) es un marco de trabajo para el gobierno y la gestión de las tecnologías de la información empresariales y dirigido a toda la empresa.
\subsection{\textcolor{teal}{Gobierno de Datos}:{Custodio de Datos (Data Steward)}}
\textbf{Definición:} Responsable de ejecutar políticas, criterios y procesos en los ámbitos de datos que le han sido asignados, basándose en la estrategia y políticas del gobierno de datos.
\subsection{\textcolor{teal}{Gobierno de Datos}:{Custodio Técnico de Datos (Technical Data Steward)}}
\textbf{Definición:} Proporciona experiencia técnica relacionada a los sistemas, procesos de extracción, transformación y carga (ETL), almacenes de datos y herramientas de inteligencia comercial.
\subsection{\textcolor{teal}{Gobierno de Datos}:{DAMA (Data Managment Association)}}
\textbf{Definición:} Asociación de Gestión de Datos es una organización mundial que promueve el entendimiento, desarrollo y prácticas relacionadas con la gestión de datos e información para soportar estrategias de negocio.

\subsection{\textcolor{teal}{Gobierno de Datos}:{Dato (Data)}}
\textbf{Definición:} Un dato es una representación simbólica de distinto tipo (numérica, de texto, espacial, etc.) de un atributo o variable cuantitativa o cualitativa. Los datos describen hechos empíricos, sucesos o entidades.
\subsection{\textcolor{teal}{Gobierno de Datos}:{Dato Personal}}
\textbf{Definición:} Cualquier información vinculada o referida a una persona natural, identificada o identificable a través de medios que puedan ser razonablemente utilizados.
\subsection{\textcolor{teal}{Gobierno de Datos}:{Datos masivos/ Macrodatos (Big Data)}}
\textbf{Definición:} Hace referencia a conjuntos de datos de gran volumen y complejidad que precisan de aplicaciones informáticas no tradicionales de procesamiento de datos para tratarlos adecuadamente.
\subsection{\textcolor{teal}{Gobierno de Datos}:{Datos personales sensibles}}
\textbf{Definición:} Aquellos datos personales que conciernen o se refieren a las características físicas o morales de una persona, tales como el origen racial, ideología, afiliación política, creencias o convicciones religiosas o filosóficas, estado de salud físico o psíquico, orientación sexual, identidad de género e identidad genética y biomédica. 
\subsection{\textcolor{teal}{Gobierno de Datos}:{Derecho de acceso (Ley de datos personales)}}
\textbf{Definición:} Derecho del titular de datos a solicitar y obtener del responsable confirmación acerca de si sus datos personales están siendo tratados por él, acceder a ellos en su caso, y a la información prevista en esta ley. 
\subsection{\textcolor{teal}{Gobierno de Datos}:{Derecho de cancelación (Ley de datos personales)}}
\textbf{Definición:} Derecho del titular de datos a solicitar y obtener del responsable que suprima o elimine sus datos personales, de acuerdo a las causales previstas en la ley. 
\subsection{\textcolor{teal}{Gobierno de Datos}:{Derecho de oposición (Ley de datos personales)}}
\textbf{Definición:} Derecho del titular de datos que se ejerce ante el responsable con el objeto de requerir que no se lleve a cabo un tratamiento de datos determinado, de conformidad a las causales previstas en la ley. 
\subsection{\textcolor{teal}{Gobierno de Datos}:{Derecho de portabilidad de los datos personales (Ley de datos personales)}}
\textbf{Definición:} Derecho del titular de datos a solicitar y obtener del responsable en un formato electrónico estructurado, genérico y de uso habitual, una copia de sus datos personales y comunicarlos o transferirlos a otro responsable de datos.
\subsection{\textcolor{teal}{Gobierno de Datos}:{Derecho de rectificación (Ley de datos personales)}}
\textbf{Definición:} Derecho del titular de datos a solicitar y obtener del responsable que modifique o complete sus datos personales, cuando están siendo tratados por él y sean inexactos o incompletos. 
\subsection{\textcolor{teal}{Gobierno de Datos}:{Derechos ARCO (Ley de datos personales)}}
\textbf{Definición:} Corresponden a un acrónimo de los derechos del titular de datos sobre sus datos. Incluye el Derecho de Acceso, de Rectificación, de Cancelación y de Oposición.
\subsection{\textcolor{teal}{Gobierno de Datos}:{Dueño de Datos (Data Owner)}}
\textbf{Definición:} El último responsable de la información asociada los datos.
\subsection{\textcolor{teal}{Gobierno de Datos}:{Encriptación de Datos (Data encryption)}}
\textbf{Definición:} La encriptación de datos es un proceso de codificación mediante el cual se altera el contenido de la información haciéndola ilegible, de esta manera se consigue mantener la confidencialidad de la información mientras se transmite del emisor al receptor.
\subsection{\textcolor{teal}{Gobierno de Datos}:{Enmascaramiento de Datos (Data Masking)}}
\textbf{Definición:} Proceso de ocultar elementos de datos en un almacenamiento. Generalmente, se utiliza para proteger un dato que es clasificado como identificador o sensible, sea de índole personal o comercial, pero a pesar de todo, debe permanecer utilizable.
\subsection{\textcolor{teal}{Gobierno de Datos}:{Escape de Datos (Data Breach)}}
\textbf{Definición:} Es una infracción de la seguridad, en la que datos sensibles, protegidos o confidenciales son copiados, transmitidos, vistos, robados o utilizados por una persona no autorizada para hacerlo.
\subsection{\textcolor{teal}{Gobierno de Datos}:{ETL (Extract, transform and load)}}
\textbf{Definición:} Es un tipo de integración de datos que hace referencia a los tres pasos (extraer, transformar, cargar) que se utilizan para mezclar datos de múltiples fuentes.
\subsection{\textcolor{teal}{Gobierno de Datos}:{Fuentes de acceso público}}
\textbf{Definición:} Todas aquellas bases de datos personales, públicas o privadas, cuyo acceso o consulta puede ser efectuado en forma lícita por cualquier persona, sin existir restricciones o impedimentos legales para su acceso o utilización.

Las dudas o controversias que se susciten sobre si una determinada base de datos es considerada fuente de acceso público serán resueltas por la Agencia de Protección de Datos Personales, quien podrá identificar categorías genéricas, clases o tipos de registros o bases de datos que posean esta condición.
\subsection{\textcolor{teal}{Gobierno de Datos}:{Gestión de Datos (Data Management)}}
\textbf{Definición:} Desarrollo y ejecución de arquitecturas, políticas, prácticas y procedimientos que gestionan apropiadamente las necesidades del ciclo de vida completo de los datos de una empresa. Se consideran las siguientes áreas:
Modelado de datos
Limpieza de datos
Administración de base de datos
Almacén de datos
Migración de datos
Minería de datos
Calidad de datos
Seguridad de datos
Gestión de meta-datos
Arquitectura de datos
\subsection{\textcolor{teal}{Gobierno de Datos}:{Gestión de Metadatos (Metadata Management)}}
\textbf{Definición:} Corresponde a la administración de datos que describen otros datos. Da significado y describe los activos de información en su organización.
\subsection{\textcolor{teal}{Gobierno de Datos}:{Gobernanza de Datos (Data Governance)}}
\textbf{Definición:} Formulación de políticas para optimizar, asegurar y aprovechar la información como un activo de la empresa, alineando objetivos de múltiples funciones.
\subsection{\textcolor{teal}{Gobierno de Datos}:{Hurto/Robo de Datos (Data Theft)}}
\textbf{Definición:} El robo de datos es el acto de robar información almacenada en bases de datos, dispositivos y servidores corporativos. Esta forma de robo corporativo es un riesgo significativo para las empresas de todos los tamaños y puede originarse tanto dentro como fuera de una organización.
\subsection{\textcolor{teal}{Gobierno de Datos}:{Información Sensible (Sensitive Information)}}
\textbf{Definición:} Corresponde a la información personal privada de un individuo, por ejemplo ciertos datos personales y bancarios, contraseñas de correo electrónico e incluso el domicilio en algunos casos. Un dato puede ser sensible dependiendo de su contexto y de los riesgos que genere su uso.
\subsection{\textcolor{teal}{Gobierno de Datos}:{Integración de Datos (Data Integration)}}
\textbf{Definición:} Proceso que implica la combinación de datos de múltiples fuentes en un solo conjunto de datos con el objetivo de proporcionar a los usuarios, acceso y entregas consistentes de datos en todo el espectro de temas y tipos de estructuras, y para satisfacer distintas necesidades de información.
\subsection{\textcolor{teal}{Gobierno de Datos}:{Inteligencia Artificial (Artificial Intelligence)}}
\textbf{Definición:} Tecnología mediante la cual podemos crear sistemas inteligentes que pueden simular la inteligencia humana.
\subsection{\textcolor{teal}{Gobierno de Datos}:{Inteligencia de Negocios (Business Intelligence) }}
\textbf{Definición:} Conjunto de estrategias, aplicaciones, datos, productos, tecnologías y arquitectura técnicas enfocados en la administración y creación de conocimiento, a través del análisis de los datos existentes en una organización o empresa.
\subsection{\textcolor{teal}{Gobierno de Datos}:{Lago de Datos (Data Lake)}}
\textbf{Definición:} Es un repositorio centralizado que permite almacenar, compartir, gobernar y descubrir grandes cantidades de datos estructurados y no estructurados de una organización a cualquier escala. Es el lugar en el que se vuelcan los datos en bruto.
\subsection{\textcolor{teal}{Gobierno de Datos}:{Limpieza de Datos (Data Cleansing)}}
\textbf{Definición:} Corresponde a la revisión, descubrimiento y corrección o eliminación de registros de datos erróneos de una tabla o base de datos. El proceso de limpieza de datos permite identificar datos incompletos, incorrectos, inexactos, no pertinentes, etc. y luego substituir, modificar o eliminar estos "datos sucios".
\subsection{\textcolor{teal}{Gobierno de Datos}:{Linaje del Dato (Data Lineage) }}
\textbf{Definición:} Corresponde al historial de los datos a lo largo de su existencia dentro de una organización. Describe el origen, características y calidad de los datos. Podría decirse que el linaje de los datos describe típicamente dónde comienza cada dato y cómo se transforma hasta lograr resultados en diferentes proyectos empresariales.
\subsection{\textcolor{teal}{Gobierno de Datos}:{Mapeo de Datos (Data Mapping)}}
\textbf{Definición:} Es el proceso de hacer coincidir campos de una base de datos con otra. Es el primer paso para facilitar la migración de datos, catalogación de datos, la integración de datos y otras tareas de gestión de datos.
\subsection{\textcolor{teal}{Gobierno de Datos}:{Metadato (Metadata)}}
\textbf{Definición:} Información que describe las características de un dato, tales como nombre, ubicación, nivel de criticidad, calidad, reglas del negocio y relación con otros datos.
\subsection{\textcolor{teal}{Gobierno de Datos}:{Migración de Datos (Data Migration)}}
\textbf{Definición:} Es el proceso de mover datos de una ubicación a otra, de un formato a otro o de una aplicación a otra. Generalmente, esto es el resultado de introducir un nuevo sistema o ubicación para los datos.
\subsection{\textcolor{teal}{Gobierno de Datos}:{Minería de datos (Data Mining)}}
\textbf{Definición:} Es el proceso de analizar grandes cantidades de datos para encontrar tendencias y patrones. Permite convertir datos sin procesar y estructurar en información comprensible sobre y para el negocio.  

\subsection{\textcolor{teal}{Gobierno de Datos}:{Modelado de Datos (Data Modeling)}}
\textbf{Definición:} Es el proceso de crear una representación visual de un sistema de información completo o partes de él para comunicar conexiones entre puntos de datos y estructuras. Su objetivo es ilustrar los tipos de datos utilizados y almacenados dentro del sistema, las relaciones entre estos tipos de datos, las formas en que se pueden agrupar y organizar los datos y sus formatos y atributos.
\subsection{\textcolor{teal}{Gobierno de Datos}:{NoSQL}}
\textbf{Definición:} Sistema de gestión de base de datos que no utiliza SQL como lenguaje de consulta principal. Las bases de datos NoSQL (“no solo SQL”) incluyen categorías como almacenes de valores clave. Estas bases de datos están optimizadas para operaciones de lectura y escritura altamente escalables en lugar de por coherencia.
\subsection{\textcolor{teal}{Gobierno de Datos}:{Nube (Cloud)}}
\textbf{Definición:} Es el uso de una red de servidores remotos conectados a internet para almacenar, administrar y procesar datos, servidores, bases de datos, redes y software. Corresponde a una estructura donde el software y el hardware están virtualmente integrados, y no hay dependencia de un servicio físico instalado.
\subsection{\textcolor{teal}{Gobierno de Datos}:{Paisaje de Datos (Data Landscape)}}
\textbf{Definición:} Un paisaje o panorama de datos es la representación de los activos de datos de una organización, las opciones de almacenamiento, los sistemas para crear, analizar, procesar y almacenar datos y otras aplicaciones presentes en el entorno de datos de una empresa. Esta información se puede ilustrar dinámica y sistemáticamente mediante una representación semántica

\subsection{\textcolor{teal}{Gobierno de Datos}:{Perfilamiento de Datos (Data Profiling)}}
\textbf{Definición:} Es el proceso de examinar los datos disponibles de una fuente de información existente, con el fin de identificar problemas y evaluar si se cumplen las condiciones mínimas para que esta información sea útil; o recopilar estadísticas o resúmenes informativos sobre esos datos.
\subsection{\textcolor{teal}{Gobierno de Datos}:{Política de Datos (Data Policy)}}
\textbf{Definición:} Instrumento que proporciona reglas, principios y pautas a modo de marco para la toma de decisiones relacionadas con el dato dentro de la empresa.
\subsection{\textcolor{teal}{Gobierno de Datos}:{Proceso de anonimización o disociación}}
\textbf{Definición:} Procedimiento en virtud del cual los datos personales no pueden asociarse al titular ni permitir su identificación, por haberse destruido el nexo con toda información que lo identifica o porque dicha asociación exige un esfuerzo no razonable, entendiendo por tal el empleo de una cantidad de tiempo, gasto o trabajos desproporcionados. Un dato anonimizado deja de ser un dato personal. 
\subsection{\textcolor{teal}{Gobierno de Datos}:{Reconocimiento / Conciencia de los Datos (Data awareness)}}
\textbf{Definición:} Es la visibilidad multidimensional de su almacenamiento. Lo que proporciona inteligencia sobre su contenido, usuarios y actividad.
\subsection{\textcolor{teal}{Gobierno de Datos}:{Repositorio de Datos (Data Repository)}}
\textbf{Definición:} Base de datos, aplicación u otra ubicación donde los datos son almacenados electrónicamente.
\subsection{\textcolor{teal}{Gobierno de Datos}:{Respaldo de Seguridad (Backup)}}
\textbf{Definición:} Corresponde a una copia de los datos originales, que se realiza con el fin de disponer de un medio para recuperarlos en caso de su pérdida.
\subsection{\textcolor{teal}{Gobierno de Datos}:{Seguridad de Datos (Data Security)}}
\textbf{Definición:} Proceso de protección de datos corporativos y prevención de la pérdida de datos por fuerzas destructivas, accesos no autorizados y otras acciones no deseadas (ej: ataque cibernético).
\subsection{\textcolor{teal}{Gobierno de Datos}:{Silo de Datos (Data Silo)}}
\textbf{Definición:} Es un repositorio independiente que aloja los datos de un determinado departamento o sección de una organización con un propósito específico. Cada uno con diferentes políticas o tecnologías. Generalmente ocurren cuando no existe un lugar o un sistema centralizado en el que almacenar todos los datos de la organización. 
\subsection{\textcolor{teal}{Gobierno de Datos}:{Sistema de Gestión de Bases de Datos (Database Management System o DBMS)}}
\textbf{Definición:} Es un software que permite administrar una base de datos. Esto significa que mediante este programa se puede utilizar, configurar y extraer información almacenada​.
\subsection{\textcolor{teal}{Gobierno de Datos}:{SQL}}
\textbf{Definición:} SQL (Structured Query Language) es un lenguaje gestor para el manejo de la información en las bases de datos relacionales. Este tipo de lenguaje de programación permite comunicarse con la base de datos y realizar operaciones de acceso y manipulación de la información almacenada.
\subsection{\textcolor{teal}{Gobierno de Datos}:{Titular de datos o titular}}
\textbf{Definición:} Persona natural, identificada o identificable, a quien conciernen o se refieren los datos personales. 
\subsection{\textcolor{teal}{Gobierno de Datos}:{Transmisión de Datos (Data Streaming)}}
\textbf{Definición:} Corresponde al proceso de transmisión de un flujo continuo de datos (también conocidos como flujos) que generalmente se introducen en el software de procesamiento de flujos para obtener información valiosa. Un flujo de datos consta de una serie de elementos de datos ordenados en el tiempo. Los datos representan un "evento" o un cambio de estado que ha ocurrido en la empresa y es útil para que la empresa los conozca y los analice, a menudo en tiempo real.
\subsection{\textcolor{teal}{Gobierno de Datos}:{Tratamiento de datos}}
\textbf{Definición:} Cualquier operación o conjunto de operaciones o procedimientos técnicos, de carácter automatizado o no, que permitan recolectar, procesar, almacenar, comunicar, transmitir o utilizar de cualquier forma los datos personales.
\section{\textcolor{blue}{Inmobiliaria}}
\subsection{\textcolor{teal}{Inmobiliaria}:{Arquitecto CABIDA proyecto}}
\textbf{Definición:} Corresponde al arquitecto asignado para realizar la CABIDA arquitectónica de un proyecto inmobiliario.
\subsection{\textcolor{teal}{Inmobiliaria}:{CABIDA}}
\textbf{Definición:} Estudio de cabida arquitectónica que permite saber que se puede construir en el terreno, coeficiente de constructibilidad, distanciamientos, coeficiente de ocupación.
\subsection{\textcolor{teal}{Inmobiliaria}:{Carta gantt proyecto}}
\textbf{Definición:} Indica las fechas relevantes en el desarrollo de un proyecto inmobiliario. Comienza con la compra de terrenos y finaliza con la entrega de los departamentos. Los hitos reflejados en la carta gantt son los siguientes: Compra terrenos, anteproyecto, propuesta de valor, desarrollo arquitectura preliminar, permisos y aprobaciones, desarrollo arquitectura, diseño de interiores, especialistas, presupuesto y aprobación directorio, inicio obras, inicio ventas, proyectos y permisos, escrituración y entrega. 
\subsection{\textcolor{teal}{Inmobiliaria}:{Categoría de proyecto}}
\textbf{Definición:} Indica en qué etapa se encuentra el proyecto. Para el sistema PSP existen 3 categorías en las cuales se clasifican los proyectos: en obra, en desarrollo y cerrado.
\subsection{\textcolor{teal}{Inmobiliaria}:{CIP}}
\textbf{Definición:} Certificado de informes previos. En este documento se indican los metros cuadrados que puede tener el proyecto, la altura máxima, distanciamientos, rasantes, etc.
\subsection{\textcolor{teal}{Inmobiliaria}:{Código PCT/ Carpeta física}}
\textbf{Definición:} Ambos conceptos hacen referencia al código distintivo de un terreno creado para la clasificación de estos mismos. Está compuesto por las primeras iniciales de la comuna, seguido del número del terreno.
\subsection{\textcolor{teal}{Inmobiliaria}:{Datos del mercado inmobiliario}}
\textbf{Definición:} Corresponde a datos relevantes de la oferta inmobiliaria y considera datos de superficie (m2), precios (UF), comunas, un periodo del año (trimestre). Esta información se encuentra actualizada a través de Real Data Consultores Inmobiliarios.
\subsection{\textcolor{teal}{Inmobiliaria}:{Datos del proyecto}}
\textbf{Definición:} Datos que hacen referencia a aspectos generales del proyecto como el nombre, alias, ubicación, estado y otros aspectos técnicos del proyecto inmobiliario.
\subsection{\textcolor{teal}{Inmobiliaria}:{Datos del terreno}}
\textbf{Definición:} Los datos del terreno incluyen información relacionada con el nombre, comuna, provincia, región, normativa, corredor de propiedades, cantidad de propiedades por terreno, entre otras.
\subsection{\textcolor{teal}{Inmobiliaria}:{Datos descripción del proyecto}}
\textbf{Definición:} Datos que hacen referencia a aspectos descriptivos de proyecto, como el número de unidades, estacionamientos, bodegas, tipologías, entre otros datos.
\subsection{\textcolor{teal}{Inmobiliaria}:{Datos Financieros del Proyecto}}
\textbf{Definición:} El resumen financiero del proyecto es presentado a directorio a través de los "temblores" del proyecto. Existen distintas versiones de estos, pero los más relevantes corresponden al temblor compra, temblor piso tipo y temblor lanzamiento.
\subsection{\textcolor{teal}{Inmobiliaria}:{EECC}}
\textbf{Definición:} Espacios Comunes
\subsection{\textcolor{teal}{Inmobiliaria}:{Estado proyecto}}
\textbf{Definición:} Se reconocen 6 estados en el desarrollo de un proyecto inmobiliario: Estado iniciado, obra en construcción, inicio de ventas, permiso edificación, planera versión 0, desarrollo arquitectura e inicio de obra.
\subsection{\textcolor{teal}{Inmobiliaria}:{Metraje departamento}}
\textbf{Definición:} Indica el número de metros cuadrados que tiene determinado tipo de departamento dentro de un proyecto inmobiliario. Por ejemplo, el departamento Tipo 1D1B tiene un metraje de 36,9 m2.
\subsection{\textcolor{teal}{Inmobiliaria}:{Mix departamento}}
\textbf{Definición:} Indica cuantas unidades hay por tipología de departamento, y su respectivo metraje. Por ejemplo, existen 35 unidades del tipo 2D1B, y cada uno de esos departamentos tiene un metraje de 50 m2.
\subsection{\textcolor{teal}{Inmobiliaria}:{ODI}}
\textbf{Definición:} Oficina Diseño de Interiores
\subsection{\textcolor{teal}{Inmobiliaria}:{Plancheta}}
\textbf{Definición:} Instrumento de topografía para levantar planos sobre el terreno que se va a utilizar para construir el proyecto inmobiliario.
\subsection{\textcolor{teal}{Inmobiliaria}:{Proyecto inmobiliario}}
\textbf{Definición:} Proyecto de obra de construcción de un inmueble, generalmente se hará referencia a este concepto con el término "proyecto".
\subsection{\textcolor{teal}{Inmobiliaria}:{Temblor}}
\textbf{Definición:} Resumen financiero proyecto
\subsection{\textcolor{teal}{Inmobiliaria}:{Temblor compra}}
\textbf{Definición:} Análisis financiero del proyecto al momento de la compra del terreno.
\subsection{\textcolor{teal}{Inmobiliaria}:{Temblor lanzamiento}}
\textbf{Definición:} Análisis financiero del proyecto al momento de empezar la venta de los proyectos.
\subsection{\textcolor{teal}{Inmobiliaria}:{Temblor piso tipo}}
\textbf{Definición:} Análisis financiero del proyecto al momento de firmar el piso tipo, una vez que está definido por los arquitectos.
\subsection{\textcolor{teal}{Inmobiliaria}:{Tipología Departamento}}
\textbf{Definición:} Indica el tipo de departamento dentro del proyecto inmobiliario. En la sigla de la tipología generalmente se encuentra estipulado el número de dormitorios y baños. Por ejemplo, 1D1B y 2D2B, para indicar 1 dormitorio y 1 baño; y 2 dormitorios y 2 baños, respectivamente.
\subsection{\textcolor{teal}{Inmobiliaria}:{Unidades departamento}}
\textbf{Definición:} Indica el número de unidades que tiene un proyecto inmobiliario. Generalmente, se considera un número de unidades asociado a una tipología de departamentos. Por ejemplo, existen 40 unidades del tipo 2D2B.
\section{\textcolor{blue}{Lean}}
\subsection{\textcolor{teal}{Lean}:{ 5 por qué}}
\textbf{Definición:} Método para investigar la causa raíz de un problema a través de preguntas. Es ir repitiendo la pregunta "¿por qué?" sobre los detalles de cada respuesta predecesora. Así se logra alcanzar la causa raíz.
\subsection{\textcolor{teal}{Lean}:{Panel de Control}}
\textbf{Definición:} Los paneles de control contienen información relevante para conocer el estado de trabajo del equipo (con cifras, indicadores y gráficos sobre la operación diaria), así como con información de las actividades y/o proyectos en los cuales el área está participando.
\subsection{\textcolor{teal}{Lean}:{SIPOC}}
\textbf{Definición:} Es un acrónimo que corresponde a las cinco partes principales de un proceso:
S (supplier): proveedor.
I (input): entrada.
P (process): proceso.
O (output): salida.
C (customer): cliente.

Los SIPOC son diagramas que se construyen de manera colaborativa (en equipo). Se hacen 5 columnas correspondientes a las 5 partes descritas y bajo ellas se listan los proveedores, las entradas, procesos, salidas y clientes para el proceso o grupo de procesos a representar.
\subsection{\textcolor{teal}{Lean}:{Sistema 5S}}
\textbf{Definición:} El sistema 5S debe su nombre a una sigla que hace referencia a cinco palabras en japonés:
1.	Seiri (separar).
2.	Seiton (organizar).
3.	Seiso (limpiar).
4.	Seiketsu (estandarizar).
5.	Shitsuke (sustentar).
Cada una corresponde a una parte fundamental de 5S y, por definición, un sistema como este no funciona si estas partes no se trabajan en conjunto. Este sistema, persigue el orden y limpieza de los espacios para hacerlos más funcionales, saludables y seguros. Las premisas de este sistema fueron cinco:
1.	Tiene que ser fácil de entender.
2.	En un contexto empresarial, debe aplicar para toda la empresa.
3.	Tiene que mejorar las condiciones de trabajo para las personas.
4.	Debe tener poca o inversión cero.
5.	Debe ser autosustentable.
\subsection{\textcolor{teal}{Lean}:{Takt Time}}
\textbf{Definición:} Es el ritmo en el que las unidades deben ser producidas para cumplir con la demanda del cliente. Debe ser considerando el tiempo y estándares que el cliente solicita.
El ajuste del Takt Time con el Lead Time busca que un proceso trabaje "justo a tiempo".
\subsection{\textcolor{teal}{Lean}:{Value Stream Mapping (VSM) / Mapeo del flujo de valor}}
\textbf{Definición:} Corresponde a un método para analizar problemas en un proceso. Primero se construye un diagrama del proceso a analizar de forma colaborativa (en equipo), para luego examinar cada una de sus actividades. Durante esta actividad, la intención es reconocer las tareas que forman parte del proceso, sus desperdicios, identificar qué actividades tienen valor agregado y cuales no, tiempo promedio para realizar cada actividad, etc.
\section{\textcolor{blue}{RRHH}}
\subsection{\textcolor{teal}{RRHH}:{Administración de la Compensación}}
\textbf{Definición:} Corresponde a una variable dicotómica que indica si el trabajador cuenta o no con administración de la compensación, en función de su contrato de trabajo y sus condiciones negociadas al ingresar a la empresa.
\subsection{\textcolor{teal}{RRHH}:{Anexo de Contrato}}
\textbf{Definición:} Es la formalidad legal que se realiza cuando existen modificaciones o se incorporan nuevos antecedentes al acuerdo contractual. Este debe quedar por escrito en un documento adicional al contrato de trabajo y debe ser de común acuerdo entre el empleador y el trabajador.
\subsection{\textcolor{teal}{RRHH}:{Área}}
\textbf{Definición:} Corresponde a una división de la gerencia. Si el área apunta a dos gerencias, se debe crear una área para cada gerencia. Por ejemplo, post venta.
\subsection{\textcolor{teal}{RRHH}:{Asignaciones}}
\textbf{Definición:} Beneficio monetario adicional a la remuneración percibido por el trabajador. Las asignaciones pueden ser no imponibles, tales como el viático, movilización y asignación familiar o imponibles como asignación zona o asignación de internet. En este último caso, se le da tratamiento como concepto bono.
\subsection{\textcolor{teal}{RRHH}:{Ausentismo}}
\textbf{Definición:} Inasistencia por causas justificadas o injustificadas al trabajo de un empleado.
\subsection{\textcolor{teal}{RRHH}:{Autoevaluación de Desempeño}}
\textbf{Definición:} Corresponde a la primera parte del proceso de evaluación de desempeño en la empresa y consiste en la valoración que hace un trabajador acerca de su propio trabajo en la empresa, de acuerdo a las competencias y conductas establecida para su perfil de cargos.
\subsection{\textcolor{teal}{RRHH}:{Base de datos personales}}
\textbf{Definición:} Conjunto organizado de datos personales, cualquiera sea la forma o modalidad de su creación, almacenamiento, organización y acceso, que permita relacionar los datos entre sí, así como realizar el tratamiento de ellos.
\subsection{\textcolor{teal}{RRHH}:{Bono}}
\textbf{Definición:} Beneficio monetario adicional a la remuneración percibido por el trabajador, corresponde a remuneración variable y puede darse en distintas periodicidades y con distintos motivos (mensual, bimensual, anual, etc.). En esta categoría también se incluyen las asignaciones imponibles (por ejemplo, asignación zona).
\subsection{\textcolor{teal}{RRHH}:{Calculadora de Cargo}}
\textbf{Definición:} Permite definir en nivel de presencialidad del cargo en base a una serie de parámetros. 
\subsection{\textcolor{teal}{RRHH}:{Capacitación laboral}}
\textbf{Definición:} Corresponde a un método aplicado por las empresas para que su personal adquiera nuevos conocimientos profesionales. La capacitación puede ser realizada tanto para que el trabajador adquiera habilidades técnicas como para la adquisición de habilidades conductuales.
\subsection{\textcolor{teal}{RRHH}:{Cargo}}
\textbf{Definición:} Objeto único asociado con la información de la posición.
\subsection{\textcolor{teal}{RRHH}:{Cargo operativo obra}}
\textbf{Definición:} Cargo que no tiene administración de la compensación y que corresponde en un 100% a un desempeño en obra, por tanto, con contrato por obra o faena.
\subsection{\textcolor{teal}{RRHH}:{Cargo sujeto a evaluación de desempeño}}
\textbf{Definición:} Corresponde al estado en el que se encuentra el cargo respecto a la evaluación de desempeño: cargo sin evaluación o cargo con evaluación.
\subsection{\textcolor{teal}{RRHH}:{Carta oferta laboral}}
\textbf{Definición:} Documento con el que una empresa comunica a un candidato que es apto para ocupar el cargo para el que postuló y en el que se especifica a modo de oferta información sobre el cargo, tipo de contrato, salario, entre otros aspectos.
\subsection{\textcolor{teal}{RRHH}:{Clima Organizacional}}
\textbf{Definición:} Conjunto de las sensaciones e impresiones de los colaboradores de una empresa sobre el ambiente laboral.
\subsection{\textcolor{teal}{RRHH}:{Comisiones}}
\textbf{Definición:} Parte de la remuneración variable por concepto de ventas.
\subsection{\textcolor{teal}{RRHH}:{Competencia crítica}}
\textbf{Definición:} Corresponde a aquella competencia deseable para el cargo, con un nivel de puntaje 4 en las conductas que definen dicha competencia. (Estas competencias pueden ser tanto transversales como específicas).
\subsection{\textcolor{teal}{RRHH}:{Competencia deseable}}
\textbf{Definición:} Corresponde a aquella competencia deseable para el cargo, con un nivel de puntaje 3 en las conductas que definen dicha competencia. (Aplica a las competencias transversales)
\subsection{\textcolor{teal}{RRHH}:{Competencias}}
\textbf{Definición:} Habilidades y aptitudes necesarias para desarrollar de mejor manera el trabajo en la empresa. Cada competencia está compuesta por 3 conductas. Las competencias pueden ser transversales o específicas.
\subsection{\textcolor{teal}{RRHH}:{Competencias específicas}}
\textbf{Definición:} Habilidades y aptitudes necesarias para desarrollar de mejor manera el trabajo. Corresponden a 2 - 3 competencias de acuerdo al perfil de cargo y a la unidad de negocio. Actualmente, Empresas Socovesa reconoce 10 competencias específicas en su modelo por competencias: Credibilidad técnica; contribución a la productividad y a la eficiencia; coordinación y control; preocupación por el impacto en el entorno y sustentabilidad; mediación y resolución de conflicto; visión de largo plazo del negocio y anticipación de nuevos escenarios; confiabilidad y seguridad; conexión permanente y gestión de redes; liderazgo participativo y desafiante; y habilidad para generar compromiso.
\subsection{\textcolor{teal}{RRHH}:{Competencias transversales}}
\textbf{Definición:} Habilidades y aptitudes necesarias para desarrollar de mejor manera el trabajo que aplica a todos los cargos de la organización. Actualmente, Empresas Socovesa reconoce 10 competencias transversales en su modelo por competencias: Curiosidad y flexibilidad mental; estructura, organización y racionalización; empatía, cooperación y trabajo colaborativo; apertura al cambio, la novedad y la diferencia; manejo de la presión y de la urgencia; análisis y resolución de problemas; cumplimiento de compromisos; comunicación efectiva, persuasión y negociación; iniciativa, agilidad y velocidad; y preocupación por el bienestar de las personas.
\subsection{\textcolor{teal}{RRHH}:{Conductas}}
\textbf{Definición:} Corresponde al comportamiento observable del trabajador dentro de cada competencia. Una conducta puede ser clasificada dentro de 5 niveles.
\subsection{\textcolor{teal}{RRHH}:{Contrato de Trabajo}}
\textbf{Definición:} El contrato de trabajo se celebra por palabra o por escrito. Corresponde a un acuerdo entre el trabajador y el empleador, por el cual el primero se compromete a prestar servicios personales bajo subordinación y dependencia de un empleador, quien se compromete a pagar una remuneración por los servicios prestados.
\subsection{\textcolor{teal}{RRHH}:{Contrato Indefinido}}
\textbf{Definición:} El contrato indefinido es aquel que se establece sin definir límites de tiempo en la prestación de los servicios, en cuanto a la duración del contrato. En la práctica, es el usado con mayor frecuencia y sólo puede terminar por despido justificado, renuncia o muerte del trabajador.
\subsection{\textcolor{teal}{RRHH}:{Contrato por Obra o Faena}}
\textbf{Definición:} El contrato por obra o faena es aquel en que el trabajador se obliga con el empleador a ejecutar una obra específica y determinada, en su inicio y en su término. La vigencia del contrato se encuentra limitada a la duración de la obra o faena.
\subsection{\textcolor{teal}{RRHH}:{Curriculum Vitae (CV)}}
\textbf{Definición:} Documento que recoge información de índole personal (datos biográficos, residencia), educativa y formativa (académica, profesional) y laboral (experiencia, habilidades y conocimientos), que un individuo ha adquirido a lo largo de su vida, con el objetivo de servirle como presentación o requisito para postularse a un puesto de trabajo.
\subsection{\textcolor{teal}{RRHH}:{Datos Académicos}}
\textbf{Definición:} Corresponde a los antecedentes curriculares del trabajador, donde se incluye nivel de estudios, profesión, institución de estudios, entre otros.
\subsection{\textcolor{teal}{RRHH}:{Datos de contacto del empleado}}
\textbf{Definición:} Corresponde al grupo de datos que permite establecer una comunicación con el trabajador. Principales datos de contacto son: celular, correo personal y dirección.
\subsection{\textcolor{teal}{RRHH}:{Datos de emergencia del empleado}}
\textbf{Definición:} Corresponde al grupo de datos que entrega información en el caso de una emergencia e incluye el contacto en caso de una emergencia, nombre de la persona, relación, número de teléfono y correo electrónico, además de la institución médica de preferencia.
\subsection{\textcolor{teal}{RRHH}:{Datos de identificación del empleado}}
\textbf{Definición:} Corresponde al conjunto de datos que se relacionan con el trabajador y los identifica, es decir, permite su identificación. Algunos de estos datos son, nombres, apellidos, sexo, estado civil, nacionalidad, RUT, etc.
\subsection{\textcolor{teal}{RRHH}:{Datos para depósito del empleado}}
\textbf{Definición:} Corresponde al conjunto de datos necesarios para efectuar un pago: tipo de remuneración, método de pago, país/región del banco, banco, tipo de cuenta, número de cuenta, propietario de cuenta, entre otros.
\subsection{\textcolor{teal}{RRHH}:{Datos previsionales del empleado}}
\textbf{Definición:} Corresponden a información de las instituciones donde es depositada la cotización obligatoria del trabajador. Generalmente hablamos de una administradora de fondos de pensiones y prestadores de salud (FONASA o Isapre). Estos datos corresponden a nombre AFP, nombre Isapre o FONASA y el plan en UF de Isapre (si es el caso).

\subsection{\textcolor{teal}{RRHH}:{Datos relación de parentesco familiar dentro de la empresa}}
\textbf{Definición:} Corresponde a la información de la declaración de relación de parentesco con otras personas de Empresas Socovesa. Los datos solicitados son: Vinculo con el familiar, nombre completo y cargo.
\subsection{\textcolor{teal}{RRHH}:{Departamento}}
\textbf{Definición:} Corresponde a una división de la gerencia y del área donde el trabajador se desempeña. Por ejemplo, maquinarias.
\subsection{\textcolor{teal}{RRHH}:{Descuentos}}
\textbf{Definición:} Corresponde a los montos descontados del sueldo imponible. Los descuentos se dividen en dos categorías, por un lado, los descuentos legales destinados al pago de AFP, comisión AFP, plan de salud, seguro de cesantía, impuesto a la renta, entre otros; y por otra parte tenemos los descuentos personales que corresponden a deducciones por solicitud de anticipo, crédito, ahorro previsional voluntario, y demás situaciones similares.
\subsection{\textcolor{teal}{RRHH}:{Desempeño Histórico}}
\textbf{Definición:} Corresponde a la evaluación de desempeño histórica de los trabajadores.
\subsection{\textcolor{teal}{RRHH}:{Dimensión de la competencia}}
\textbf{Definición:} Corresponde al área de la cual se desprenden las competencias. Empresas Socovesa reconoce las siguientes dimensiones de competencias: pensamiento, cumplimiento, interacción, adaptabilidad, gestión emocional.

\subsection{\textcolor{teal}{RRHH}:{Elemento PEP}}
\textbf{Definición:} Código asignado a personas de obra para distribuir costos. El código considera la empresa, proyecto y cargo.
\subsection{\textcolor{teal}{RRHH}:{Empresa}}
\textbf{Definición:} Razón social de la unidad legal que contrata al trabajador. El conjunto de estas unidades legales, corresponde a Empresas Socovesa.
\subsection{\textcolor{teal}{RRHH}:{Estructura Organizacional}}
\textbf{Definición:} La estructura organizacional es el sistema jerárquico escogido para organizar a los trabajadores en un organigrama.
\subsection{\textcolor{teal}{RRHH}:{Evaluación de Desempeño}}
\textbf{Definición:} Corresponde al proceso que mide de forma objetiva e integral la conducta profesional, las competencias, el rendimiento y la productividad. 
\subsection{\textcolor{teal}{RRHH}:{Familia de cargo KPI}}
\textbf{Definición:} Corresponde a una herramienta de la organización para agrupar cargos similares en agrupaciones de cargos, de acuerdo a una metodología de competencias.
\subsection{\textcolor{teal}{RRHH}:{Familia de cargo Memoria}}
\textbf{Definición:} Corresponde a una herramienta de la organización para agrupar cargos similares en agrupaciones de cargos, de acuerdo a lo solicitado por la Comisión para el Mercado (CMF) para la publicación de la Memoria Anual. 
\subsection{\textcolor{teal}{RRHH}:{Feriado legal/ Vacaciones}}
\textbf{Definición:} Corresponde al descanso laboral con pago de remuneración que tiene todo trabajador. Todo trabajador que cumpla una anualidad, tiene derecho a 15 días hábiles de feriado legal.
\subsection{\textcolor{teal}{RRHH}:{Ficha de Ingreso del Trabajador}}
\textbf{Definición:} Corresponde a la ficha de incorporación solicitada por el área de recursos humanos al trabajador para la elaboración del contrato de trabajo. En ella se solicitan datos personales, datos de contacto, datos académicos, contacto de emergencia, datos para depósito, datos previsionales, entre otros.
\subsection{\textcolor{teal}{RRHH}:{Firma Electrónica}}
\textbf{Definición:} Es el equivalente electrónico de una firma manuscrita, donde a través de un conjunto de datos digitales se identifica al firmante de un documento electrónico y vincula la identidad del usuario con el documento firmado. Empresas Socovesa utiliza la Plataforma 5.dec para la firma electrónica de documentos.
\subsection{\textcolor{teal}{RRHH}:{Fortaleza (competencias)}}
\textbf{Definición:} Una fortaleza en competencias corresponde a un puntaje mayor al deseado o al establecido en el perfil meta.

\subsection{\textcolor{teal}{RRHH}:{Fuero maternal}}
\textbf{Definición:} Las mujeres tienen el derecho de fuero maternal, el cual protege los beneficios de la maternidad. Este, consiste en la imposibilidad del empleador de la trabajadora, de despedirla o poner fin a la relación laboral (sin autorización judicial previa), por un periodo que inicia desde la concepción hasta que el niño o niña cumpla un año y 84 días de edad.
\subsection{\textcolor{teal}{RRHH}:{Gerencia}}
\textbf{Definición:} Corresponde a la sección de la empresa donde el trabajado se desempeña. Por ejemplo, Gerencia de Sistemas.
\subsection{\textcolor{teal}{RRHH}:{Gratificación legal}}
\textbf{Definición:} Corresponde a la parte de las utilidades con que el empleador beneficia el sueldo del trabajador.
La gratificación se entrega cuando el empleador obtiene utilidades líquidas en su giro, por lo que tiene la obligación de entregar una gratificación anual a sus trabajadores. Esta gratificación puede pagarse de dos formas: un pago que no debe ser inferior a la proporción del 30% de la utilidad o se debe pagar al trabajador el 25% de lo devengado por el concepto de remuneración mensual con un tope de 4,75 ingresos mínimos mensuales.

\subsection{\textcolor{teal}{RRHH}:{Grupo Familiar}}
\textbf{Definición:} Conjunto de personas unidas o no por vínculos de sangre que dan forma al núcleo familiar.
\subsection{\textcolor{teal}{RRHH}:{GX}}
\textbf{Definición:} Grupo ejecutivo
\subsection{\textcolor{teal}{RRHH}:{Haberes}}
\textbf{Definición:} Corresponde a todos los montos a los cuales luego se les aplican los descuentos mensuales que corresponden en cada caso. En los haberes de la liquidación mensual se encuentra el sueldo imponible, los bonos, la gratificación legal y las horas extra.
\subsection{\textcolor{teal}{RRHH}:{Habilidades conductuales-relacionales a capacitar (Resultado evaluación de desempeño)}}
\textbf{Definición:} Hace referencia a las características de comportamiento que fortalecen diversos aspectos de su desempeño laboral. Corresponde a una necesidad de capacitación obtenida desde los resultados de la evaluación de desempeño.
\subsection{\textcolor{teal}{RRHH}:{Habilidades técnicas a capacitar (Resultado evaluación de desempeño)}}
\textbf{Definición:} Hace referencia a las competencias asociadas al uso de herramientas que son necesarias para ejercer el trabajo. Corresponde a una necesidad de capacitación obtenidas desde los resultados de la evaluación de desempeño.
\subsection{\textcolor{teal}{RRHH}:{Horas extraordinarias}}
\textbf{Definición:} Para calcular el valor de las horas extras, en el caso de un trabajador contratado por 45 horas semanales y con sueldo mensual, debe dividirse su sueldo en 30 y luego multiplicarse por 28, debiendo dividirse el resultado por 180, obteniéndose así el valor de cada hora ordinaria.
La cantidad máxima de horas extraordinarias (sobretiempo) a realizar en el sexto día, correspondiente al día sábado, dependerá de si el trabajador durante la semana comprendida entre lunes a viernes las ha realizado. El límite máximo diario de las horas extraordinarias es de 2 horas y su máximo semanal corresponde a 12 horas, si se considera que la jornada semanal puede distribuirse en un máximo de 6 días de labor (6 días x 2 horas extras= 12 horas extras semanales). 
\subsection{\textcolor{teal}{RRHH}:{Indicadores de eficacia capacitación mensual/anual}}
\textbf{Definición:} Corresponde al detalle de la participación en capacitación respecto al número de trabajadores, las horas de capacitación, asistencia y la aprobación de los cursos. Algunos de los indicadores son las participaciones aprobadas, participaciones reprobadas, número de frecuencia capacitación, número de inscripciones, entre otros.
\subsection{\textcolor{teal}{RRHH}:{Indicadores económicos capacitación mensual/anual}}
\textbf{Definición:} Corresponden a la inversión mensual o anual reportada en el informe de capacitación, que incluye la inversión en franquicia tributaria, inversión costo empresa y la inversión total en cursos de capacitación.
\subsection{\textcolor{teal}{RRHH}:{InfoNómina}}
\textbf{Definición:} Informe en excel obtenido del Sistema Uno con información de la renta de todos los empleados de la empresa por mes.
\subsection{\textcolor{teal}{RRHH}:{Información aula virtual GESOC}}
\textbf{Definición:} Corresponde a la plataforma donde se realizan los cursos, cuáles son los cursos y la información de éstos impartidos por la OTEC GESOC de Empresas Socovesa. 
\subsection{\textcolor{teal}{RRHH}:{Información cursos }}
\textbf{Definición:} Corresponde información de los cursos, e incluye información de identificación de la OTEC, del curso impartido, costos, requisitos, asistencia, entre otros.
\subsection{\textcolor{teal}{RRHH}:{Información laboral}}
\textbf{Definición:} Corresponde a información laboral del trabajador que da cuenta de su condición en la empresa. Incluye los siguientes datos: fecha inicio de contrato, lugar de trabajo, ciudad, proyecto, tipo de jornada y antigüedad en la posición.
\subsection{\textcolor{teal}{RRHH}:{Información proveedores de capacitación (OTEC)}}
\textbf{Definición:} Corresponde a información de los organismos técnicos de capacitación (OTEC) que han prestado servicios a Socovesa. Los indicadores incluyen: Ranking OTEC, participaciones por OTEC, participaciones acumuladas y cantidad de horas por curso.
\subsection{\textcolor{teal}{RRHH}:{Jefatura}}
\textbf{Definición:} Corresponde a la persona encargada del equipo de trabajo en el cual el trabajador desempeña. Una jefatura directa corresponde a la jefatura inmediata del trabajador.
\subsection{\textcolor{teal}{RRHH}:{Jefe Evaluador}}
\textbf{Definición:} Corresponde a la jefatura que realizará la evaluación de desempeño del trabajador.
\subsection{\textcolor{teal}{RRHH}:{Licencia Médica}}
\textbf{Definición:} Documento extendido por un profesional de la salud, que acredita que un trabajador se encuentra incapacitado temporalmente para trabajar, e indica una cantidad de días de reposo.
\subsection{\textcolor{teal}{RRHH}:{Licencia médica postnatal}}
\textbf{Definición:} Permiso médico a mujeres embarazadas, con posterioridad a la fecha de parto. En Chile corresponde a 12 semanas (84 días) contadas desde el nacimiento del/la hijo/a. El inicio de este deberá comprobarse con el correspondiente certificado de parto.
\subsection{\textcolor{teal}{RRHH}:{Licencia médica prenatal}}
\textbf{Definición:} Permiso médico a mujeres embarazadas, previo a la fecha de parto. En Chile corresponde a seis semanas (42 días) anteriores a la fecha probable de parto.
\subsection{\textcolor{teal}{RRHH}:{Liquidación de Sueldo}}
\textbf{Definición:} Documento que el empleador entrega a sus empleados mes a mes, donde se deja constancia del pago de sueldo correspondiente y se indican los montos, las deducciones, cargas familiares, entre otros conceptos.
\subsection{\textcolor{teal}{RRHH}:{Marca}}
\textbf{Definición:} Corresponde a las distintas identificaciones comerciales dentro de Empresas Socovesa, por ejemplo: Almagro, Pilares, Socovesa Santiago, etc.
\subsection{\textcolor{teal}{RRHH}:{Matriz de talento (Nine box)}}
\textbf{Definición:} Corresponde a una matriz donde a partir de las competencias y el desempeño es posible determinar el potencial de un trabajador en la empresa. La matriz de talento cuenta con 9 niveles o cuadrantes: bajo rendimiento, desempeño sólido, desempeño sobre saliente, nuevo rol, colaborador clave, líder emergente, talento mal ubicado, listo para nuevas oportunidades y futuros líderes.
\subsection{\textcolor{teal}{RRHH}:{Modelo de gestión por competencias}}
\textbf{Definición:} Es un modelo de gestión que se basa en competencias y conductas laborales requeridos por parte de una organización que permiten un buen desempeño organizacional.
Dentro del modelo es crítico establecer los distintos niveles para cada competencia, que son los que determinarán el grado de adecuación en el que se encuentra cada empleado evaluado y que a su vez nos permitirán saber si existen diferencias entre lo requerido para desempeñar óptimamente su trabajo y la realidad, así como definir planes de capacitación y desarrollo para cubrirlos a través de distintas fórmulas.
\subsection{\textcolor{teal}{RRHH}:{Necesidades de capacitación (Resultado evaluación de desempeño)}}
\textbf{Definición:} Corresponden a las necesidades de capacitación obtenidas desde los resultados de la evaluación de desempeño e incluyen las habilidades técnicas y las habilidades conductuales o relacionales a capacitar descritas por el jefe evaluador.
\subsection{\textcolor{teal}{RRHH}:{Nivel de conducta}}
\textbf{Definición:} Corresponde a la escala para evaluar una conducta. Se distinguen 5 niveles de conducta: 
Nivel 1, ausencia de la conducta y dificultad para entenderla;
Nivel 2, entendimiento y validación de la necesidad de la conducta, dificultad para ejercerla; 
Nivel 3, manejo de la conducta a nivel individual y básica; 
Nivel 4, manejo (y promoción) de la conducta con impacto en el área directa; 
y Nivel 5, excelencia del manejo de la conducta, referente identificado por sus pares y por la organización.
\subsection{\textcolor{teal}{RRHH}:{Nivel de talento}}
\textbf{Definición:} Corresponde al potencial de un trabajador obtenido a partir de su evaluación de desempeño y se visualiza a través de la matriz de talento. Se distinguen 9 niveles: bajo rendimiento, desempeño sólido, desempeño sobre saliente, nuevo rol, colaborador clave, líder emergente, talento mal ubicado, listo para nuevas oportunidades y futuros líderes.
\subsection{\textcolor{teal}{RRHH}:{Nivel HAY}}
\textbf{Definición:} Sistema para la medición de un cargo en base a puntajes en tres dimensiones relacionadas con la obtención de resultados de un puesto: actuar, responsabilidad y saber.
\subsection{\textcolor{teal}{RRHH}:{Oportunidad de mejora (competencias)}}
\textbf{Definición:} Una brecha en competencias corresponde a un puntaje de competencia menor al deseado o al establecido en el perfil meta.
\subsection{\textcolor{teal}{RRHH}:{Organigrama}}
\textbf{Definición:} Corresponde a una representación gráfica de la estructura interna de la empresa. Sistema Uno permite visualizar el organigrama de dos maneras: mediante posiciones y mediante cargos. Datos incluidos en el organigrama son: nombre, cargo, jefaturas y subalternos.
\subsection{\textcolor{teal}{RRHH}:{Organismo técnico de capacitación (OTEC)}}
\textbf{Definición:} Empresa autorizada para diseñar y desarrollar actividades de capacitación o cursos con franquicia tributaria. Permite solicitar la autorización y acreditación al Servicio Nacional de Capacitación y Empleo (SENCE)
\subsection{\textcolor{teal}{RRHH}:{Perfil de cargo}}
\textbf{Definición:} Corresponde al grupo de cargos con un mismo nivel conductual y para una misma unidad de negocio.
\subsection{\textcolor{teal}{RRHH}:{Perfil de línea conductual}}
\textbf{Definición:} Existen 3 perfiles de línea conductual diferentes: Obra, Comercial y Corporativo, considerando líneas de conductas similares.
\subsection{\textcolor{teal}{RRHH}:{Perfil meta}}
\textbf{Definición:} Corresponde al puntaje deseado de cada competencia para determinado grupo de cargos. La diferencia entre el puntaje de la competencia y el perfil meta corresponde en una brecha o una fortaleza.
\subsection{\textcolor{teal}{RRHH}:{Persona sujeta a evaluación}}
\textbf{Definición:} Indica si la persona está sujeta a evaluación de desempeño. Considera aspectos como el ausentismo o la fecha de ingreso.
\subsection{\textcolor{teal}{RRHH}:{Personal activo}}
\textbf{Definición:} Persona que se encuentra trabajando de forma activa en la empresa.
\subsection{\textcolor{teal}{RRHH}:{Personal directivo obra}}
\textbf{Definición:} Cargo con contrato indefinido y con administración de la compensación que se desempeña en obras principalmente.
\subsection{\textcolor{teal}{RRHH}:{Personal inactivo}}
\textbf{Definición:} Persona que no se encuentra trabajando de forma activa en la empresa. Personas finiquitadas.
\subsection{\textcolor{teal}{RRHH}:{Plataforma 5.dec}}
\textbf{Definición:} Plataforma que permite la firma electrónica de documentos a los empleados y posterior a ello, su almacenamiento. Se utiliza para contratos, anexos de contratos y otros documentos. 
Enlace: https://5.dec.cl/
\subsection{\textcolor{teal}{RRHH}:{Posición}}
\textbf{Definición:} Objeto relacionado con la estructura organizacional y asignado a una persona
\subsection{\textcolor{teal}{RRHH}:{Puntaje Evaluación de Desempeño}}
\textbf{Definición:} Corresponde al puntaje final del proceso de evaluación de desempeño. La escala de dicha evaluación va de 1 a 5 (insatisfactorio, necesita mejorar, cumple expectativas, excede las expectativas, desempeño destacado)
\subsection{\textcolor{teal}{RRHH}:{Puntaje Evaluación de Desempeño Calibrado}}
\textbf{Definición:} Corresponde al puntaje de evaluación de desempeño posterior a la calibración de puntajes, proceso en el cual se ponen en revisión las evaluaciones.
\subsection{\textcolor{teal}{RRHH}:{Puntaje responsabilidad para el cargo (Metodología HAY)}}
\textbf{Definición:} Corresponde al puntaje que el cargo requiere en cuanto a la responsabilidad (account ability), de acuerdo al "Sistema HAY".
\subsection{\textcolor{teal}{RRHH}:{Puntaje saber para el cargo (Metodología HAY)}}
\textbf{Definición:} Corresponde al puntaje que el cargo requiere en cuanto al conocimiento (know how), de acuerdo al "Sistema HAY".
\subsection{\textcolor{teal}{RRHH}:{Puntaje solución de problemas para el cargo (Metodología HAY)}}
\textbf{Definición:} Corresponde al puntaje que el cargo requiere en cuanto al actuar (solución de problemas), de acuerdo al "Sistema HAY".
\subsection{\textcolor{teal}{RRHH}:{Remuneración}}
\textbf{Definición:} Pago de un servicio o trabajo establecido en el contrato de trabajo. La remuneración se compone de una parte fija (remuneración fija) y de una parte variable (remuneración variable). En general cuando hablamos de remuneración, podemos considerar cuatro grupos de conceptos:
Remuneración fija y asignaciones como parte de la renta fija, obligaciones legales (horas extras), bonos y beneficios de la empresa.
\subsection{\textcolor{teal}{RRHH}:{Remuneración fija}}
\textbf{Definición:} El porcentaje de la remuneración que corresponde a remuneración fija está determinado por la estructura de la renta y corresponde a los conceptos de pago que no varían mes a mes. Ejemplo, sueldo base y gratificación.
\subsection{\textcolor{teal}{RRHH}:{Remuneración variable}}
\textbf{Definición:} El porcentaje de la remuneración que corresponde a remuneración variable está determinado por la estructura de la renta y corresponde a los conceptos de pago que varían mes a mes. Por ejemplo, los bonos.
\subsection{\textcolor{teal}{RRHH}:{Retroalimentación laboral (Feedback)}}
\textbf{Definición:} Corresponde a la retroalimentación que le entrega la jefatura a su reporte directo en torno a su desempeño en base a su perfil de competencias.
\subsection{\textcolor{teal}{RRHH}:{Saldos de Vacaciones}}
\textbf{Definición:} Corresponde al número disponible de días que puede disponer el trabajador con motivo de feriado legal / periodo de vacaciones.
\subsection{\textcolor{teal}{RRHH}:{Sistema Uno}}
\textbf{Definición:} Plataforma SuccesFactors utilizada por el Área de Recursos Humanos para gestionar la información de los empleados. En ella se encuentra la información personal, organizativa, de remuneraciones, entre otros antecedentes.
\subsection{\textcolor{teal}{RRHH}:{Solicitudes de formación}}
\textbf{Definición:} Corresponden a las solicitudes de jefatura o del propio trabajador para la realización de cursos, diplomados o estudios superiores.
\subsection{\textcolor{teal}{RRHH}:{Subordinado}}
\textbf{Definición:} Corresponde al trabajador o trabajadores que se encuentran bajo las órdenes o la responsabilidad de una jefatura.
\subsection{\textcolor{teal}{RRHH}:{Sueldo Imponible}}
\textbf{Definición:} Corresponde a la remuneración sobre la cual el trabajador hace los descuentos de previsión y salud que correspondan.
\subsection{\textcolor{teal}{RRHH}:{Sueldo Líquido}}
\textbf{Definición:} Corresponde al sueldo que se recibe a fin de mes al que se aplicaron todos los descuentos legales pertinentes como AFP, Salud, Seguro de cesantía e impuestos.
\subsection{\textcolor{teal}{RRHH}:{SUMASU}}
\textbf{Definición:} Costo empresa del pago de la remuneración, compuesto por el total haberes (sueldo base, gratificación y movilización), seguro de invalidez y sobrevivencia, seguro de cesantía y pago de mutual.
\subsection{\textcolor{teal}{RRHH}:{Teletrabajo}}
\textbf{Definición:} El teletrabajo permite a los empleados llevar a cabo sus deberes y sus responsabilidades desde una ubicación fuera del lugar de trabajo oficial. Podría implicar trabajar desde la oficina en casa, otra sucursal, una cafetería, una librería o incluso un espacio de coworking.
De acuerdo a la definición del Código de Trabajo (DT), se considera teletrabajo cuando los servicios son prestados mediante la utilización de medios tecnológicos, informáticos o de telecomunicaciones o si tales servicios deben reportarse mediante estos medios.

\subsection{\textcolor{teal}{RRHH}:{Tipo de Contrato}}
\textbf{Definición:} Corresponde a la modalidad de trabajo bajo la cual se desempeña el trabajador. En Empresas Socovesa, principalmente existen dos: contrato indefinido y contrato por obra o faena.
\subsection{\textcolor{teal}{RRHH}:{Total Imponible}}
\textbf{Definición:} Corresponde al monto que se utiliza para calcular los valores destinados a las Administradoras de Fondos de Pensiones (AFP), salud y seguro de cesantía.
\subsection{\textcolor{teal}{RRHH}:{Total Tributable}}
\textbf{Definición:} Corresponde a los haberes imponibles menos los descuentos legales. Por lo tanto, al definir ese monto se puede calcular el impuesto que se corresponde pagar por el sueldo recibido.
\subsection{\textcolor{teal}{RRHH}:{UEN}}
\textbf{Definición:} Corresponde a la unidad de negocio definida por la actividad económica principal. Por ejemplo, desarrollo inmobiliario.
\section{\textcolor{blue}{Sistemas TI}}
\subsection{\textcolor{teal}{Sistemas TI}:{Agile-Agilidad}}
\textbf{Definición:} Termino en Desarrollo de Software que representa principios y valores que guían el desarrollo de soluciones agiles. Entre los valores /4) están: Individuos e interacciones sobre procesos y herramientas, Software que funcione sobre documentación exhaustiva, Colaboración con el cliente sobre contratos, Responder al cambio sobre el seguimiento de un plan. Entre los principios (12): Satisfacción al cliente, Bienvenidos los cambios, Entrega de Producto frecuente, Colaboración continua, Equipo motivados, Comunicación cara a cara, Avence en función de software funcionando, Promover un ritmo sostenido, Excelencia, Simplicidad, Equipos auto-organizados , Mejora continua
\subsection{\textcolor{teal}{Sistemas TI}:{DevOps}}
\textbf{Definición:} DevOps es un término usado para una metodología que  facilita el desarrollo y el despliegue a producción de forma rápida y continua. Viene de la contracción Developer y Operations. L aidea es que las áreas de desarrollo y operaciones trabajan cómo un equipo para facilitar pruebas y finalmente dejar las aplicaciones en producción. Normalmente incluye desarrollo, pruebas automáticas, y paso a prucción.
\end{document}
